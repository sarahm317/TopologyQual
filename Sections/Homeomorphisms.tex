\documentclass[../TopologyQualSolutions.tex]{subfiles}

\begin{document}

\section{Homeomorphic Spaces}

\begin{prob}{F08.7}{F08.7}
    Let $\C$ be the set of complex numbers with the standard Euclidean topology. Define $\sim$ on $\C$ by $w \sim z$ if and only if $(z-w)$ is real. Prove that $\C/\sim$ is homeomorphic to $\R$ with the standard topology.
\end{prob}

\begin{proof}
    Let $X = \R$ and $Y = \C/\sim$ and define $f: X \to Y$ be $f(a) = a + ai$. Define $g: Y \to X$ by $g(a + bi) = b$. To see that $g$ is well-defined, suppose that $z = (a + bi) \sim (c + di) = w$ in $\C/\sim$. Then $b - d = 0$ since $z - w \in \R$. Therefore $g(a+bi) = b = d = g(c + di)$, as desired. Also notice that $g\circ f = \ident_X$ and $f \circ g = \ident_Y$, proving that $f$ and $g$ are inverses. It remains to show that both $f$ and $g$ are continuous.\\

    \textcolor{red}{What is the best way to show continuity here?}

    Let $\epsilon > 0$, $x \in \R$, and consider the open ball $B_\epsilon(x) \sq \R$. 
\end{proof}

\begin{prob}{F13}{F13.7}
    Let $f: X \to Y$ be a continuous, surjective map between compact, Hausdorff spaces. Define an equivalence relation $\sim$ on $X$ so that $f$ factors as 
        \[X \xra{q} X' \xra{f'} Y\]
    where $X' = X/\sim$, $q$ is the quotient map, and $f'$ is any bijection. Prove that $f'$ is a homeomorphism.
\end{prob}

\begin{proof}
    Observe that the quotient of a compact space is compact. Therefore, $f': X/\sim \to Y$ is a map from a compact space to a Hausdorff space. Because $f'$ is a bijection, proving that $f'$ is continuous will imply that $f'$ is a homeomorphism. By definition of the quotient topology, a set in $X/\sim$ is open if and only if its preimage under $q$ is open in $X$. If $U \sq Y$ is any open set,
        \[f\inv(U) = (f' \circ q)\inv(U) = q\inv\left((f')\inv(U)\right). \]
    Since $f$ is continuous, $f\inv(U)$ is open and therefore $(f')\inv(U)$ is open. That is, $f'$ is continuous.
\end{proof}

\begin{prob}{S20}{}
Prove that $S^2$ is homeomorphic to a quotient space of $S^1 \times [0,1]$.
\end{prob}

\begin{proof}
Define an equivalence relation $\sim$ on $S^1 \times [0,1]$ such that
    \[(\theta,0) \sim (\theta', 0) \]
and
    \[(\theta,1) \sim (\theta', 1) \]
for any $\theta,\theta' \in S^1$. Then $S^1 \times [0,1]/ \sim$ is an annulus with each of the boundary disks crushed to a point. Note that
    \[S^2 = \{(\theta, \phi): 0 \leq \theta \leq 2\pi, 0 \leq \varphi \leq \pi \}. \]
where all points of the form $(\theta, 0)$ correspond to the north pole of $S^2$ and all points of the form $(\theta, \pi)$ correspond to the south pole of $S^2$. Every other point in $S^2$ has a unique description in this coordinate system.\\

Define $f: S^1 \times [0,1]/ \sim \to S^2$ by $f(\theta,t) = (\theta, \pi t)$. Observe that $f$ is well-defined as all points in $S^1 \times \{0\}$ are mapped to the north pole and all points in $S^1 \times \{1\}$ are mapped to the south pole. As both component functions of $f$ are continuous, $f$ is continuous. Given any $(\theta, \varphi) \in S^2$, $f(\theta, \varphi/\pi) = (\theta, \varphi)$, proving that $f$ is surjective. To see that $f$ is injective, suppose that $f(\theta,t) = f(\theta', t')$. Then, $(\theta,\pi t) = (\theta',\pi t')$. This means that $t = t'$. If $t = 0$, then $(\theta,0) \sim (\theta' 0)$. If $t = 1$, $(\theta,1) \sim (\theta' 1)$. If $t,t' \not\in \{0,\pi\}$ then $\theta = \theta'$. In any case, $(\theta,t) = (\theta, t') \in S^1 \times [0,1] /\sim$. As $f$ is a continuous bijection from a compact space to a Hausdorff space, $f$ is a homeomorphism.
\end{proof}

\end{document}