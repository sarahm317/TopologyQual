\documentclass[../TopologyQualSolutions.tex]{subfiles}

\begin{document}

\section{Basic Point Set Topology}

\begin{prob}{(F12.4)}{F12.4}
    Let $X,Y$ be non-empty topological spaces.
    \begin{enumerate}[(a)]
        \item Define the product topology on $X \times Y$.
        \item Define path connected.
        \item Show that $X$ and $Y$ are path connected if and only if $X \times Y$ is path connected.
    \end{enumerate}
\end{prob}

See \ref{prob:S12.2} for the definition.

A topological space $X$ is path connected if for any two points $x,y \in X$, there exists a continuous function $\gamma: [0,1] \to X$ where $\gamma(0) = x$ and $\gamma(1) = y$. Here, $\gamma$ is a path.

\begin{proof}
    See (\ref{thm:productResults}).
\end{proof}

\begin{prob}{(S20.4)}{S20.4}
    A topological space is \emph{separable} if it has a countable dense subset. Prove that the product of a countable collection of separable topological spaces is separable.
    \end{prob}
    
    \begin{proof}
        Let $\{X_n\}$ be a countable collection of separable topological spaces and let $X = \prod_{n=1}^\infty X_n$. For each $n \in \N$, let $D_n$ be a countable dense subset of $X_n$ and choose any $x_n \in D_n$. Next define
            \[U_1 = \prod_{k=1}^\infty \{x_k\}\]
        and note that $U_1$ is countable as it is the countable product of singletons. Define
            \[U_2 = D_1 \times \prod_{k=2}^\infty \{x_k\}\] 
        and note that $U_2$ is countable since it is the finite product of countable sets. Continue in this manner, defining
            \[U_n = \prod_{k=1}^{n-1} D_n \times \prod_{k=n}^\infty \{x_k\}\]
        for every $n = 2,3,\ldots$. Each $U_n$ is the finite product of countable sets and therefore is countable. Therefore, $U = \bigcup_{n=1}^\infty U_n$ is countable since it is the countable union of countable sets.
    
        \begin{claim}
            $U$ is dense in $X$.
    
            \begin{proof}
                Let $V \sq X$ be an arbitrary open set. Let $\prod_{n=1}^\infty V_n$ be a basic open set contained in $V$. Then, each $V_i \sq X_i$ is open and for all but finitely many $i$, $V_i = X_i$. For $i \in \N$ where $V_i = X_i$, $x_i \in U_i \sq U$. That is, $V_i \cap U_i \neq \varnothing$.\\
                
                Now consider $i \in \N$ where $V_i \sq X_i$. Then $V_i$ is open in $X_i$ and since $U_i$ is dense in $X_i$, there exists $u_i \in U_i \cap V_i$.\\
    
                Define a point $(a_i) \in X$ by letting $a_i = x_i$ when $V_i = X_i$ and letting $a_i = u_i$ when $V_i \sq X_i$ is open. By construction, $(a_i) \in \prod_{n=1}^\infty V_n \cap U \sq V \cap U$.
            \end{proof}
    
        \end{claim}
    \end{proof}

\begin{prob}{(S19.1)}{S19.1}
    Let $A$ and $B$ be disjoint compact subspaces of a Hausdorff topological space $X$. Prove that there are disjoint open sets $U$ and $V$ with $A \sq U$ and $B \sq V$.
\end{prob}

\begin{proof}
    See \ref{thm:compactHaussNormal}.
\end{proof}


\begin{prob}{(F17.2)}{F17.2}
    Prove or provide a counter-example to the following:

    \begin{enumerate}[(a)]
        \item A closed and bounded subset of a topological space is compact.
        \item The image of a closed subset under a continuous map is closed.
        \item If $f: X \to Y$ is a continuous surjection and $Y$ is Hausdorff then so is $X$.
        \item If $f: X \to Y$ is a continuous surjection and $X$ is Hausdorff then so is $Y$.
        \item If a function between Hausdorff topological spaces is continuous, then the preimage of every compact set is compact.
        \item If $f: X \to Y$ is a continuous injection and $Y$ is Hausdorff then so is $X$.
        \item If $Y \sq \R^2$ and $Y$ is path connected, then the closure of $Y$ is path connected.
    \end{enumerate}
\end{prob}

\begin{solution}
    Consider $\R$ with the topology induced by the discrete metric:
        \[d(x,y) = \begin{cases} 0 & x = y\\ 1 & x\neq y \end{cases}.\]
    Then $\R$ is bounded since all points are within distance 1 from the origin. That is, $\R = \ol{B_1(0)}$. However, the open cover $\{B_{1/2}(x)\}_{x \in \R}$ has no finite subcover since each ball contains exactly one point of $\R$. With respect to this metric, $\R$ is closed and bounded, but is not compact.
\end{solution}

\begin{solution}
    This is false. Let $X = [0,1]$ with the discrete metric and $Y = [0,1]$ with the indiscrete metric. Then the identity map $f: X \to Y$ is a continuous surjection, but $f(\{1/2\}) = \{1/2\}$ is not closed in $Y$.
\end{solution}

\begin{solution}
Let $X = \{0,1\}$ with the topology $\{\varnothing, \{0\}, X\}$ and $Y = \{0\}$ with the topology $\{\varnothing, Y\}$. Let $f: X \to Y$ be the zero map. Then $f$ is continuous and a surjection. Any space with a single point is trivially Hausdorff. However, $X$ is not Hausdorff as 0 and 1 cannot be separated with open sets. 
\end{solution}

\begin{solution}
    Let $X = [0,2]$ and $A = (1,2]$, with the usual topology on $\R$. Let $Y = X/A$ and $q: X \to Y$ the quotient map. Then $q$ is continuous and surjective, $X$ is Hausdorff, but $Y$ is not Hausdorff as there is no way to separate 1 from 2.
\end{solution}

\begin{solution}
    Let $f: (0,1) \to [0,1]$ be the identity map where both spaces are equipped with the subspace topology of $\R$. Clearly $f$ is continuous. But, $[0,1]$ is compact and $f\inv([0,1]) = (0,1)$ is not compact.
\end{solution}

\begin{proof}
    Let $f: X \to Y$ be a continuous injection and suppose that $Y$ is Hausdorff. Let $x, y \in X$ be distinct points. Since $f$ is an injection, $f(x) \neq f(y)$. Because $Y$ is Hausdorff, there exist disjoint open sets $U$ and $V$ that contain $f(x)$ and $f(y)$, respectively. Then, since $f$ is continuous, $f\inv(U)$ and $f\inv(V)$ are open neighborhoods of $x$ and $y$ respectively. If there were some $z \in f\inv(U) \cap f\inv(V)$, then $f(z) \in U \cap V$, which is impossible. Therefore, $f\inv(U)$ and $f\inv(V)$ separate $x$ and $y$, proving that $X$ is Hausdorff.
\end{proof}

\begin{solution}
    Consider the Topologist's Sine Curve: let $S$ be the set of points
        \[S = \{(x,\sin(1/x)): x \in (0,1]\}.\]
    The closure of $S$ is $S \cup \{0\} \times [-1,1]$. By construction $S$ is path-connected, but $\cl{S}$ is the Topologist's Sine curve and is not path-connected. See \ref{thm:topSine} for the details.
\end{solution}

\begin{prob}{(S00.4), (S13.7), (F16.1), (F17.4), (F19.2)}{S00.4}
    Define what it means for a collection of subsets of a set $X$ to be a basis for a topology on $X$. Give a necessary condition for a collection of sets to be a basis for a topology.\\

    Let $X$ be the set of subsets of $\N$. If $A$ is a finite subset of $\N$ and $B \sq \N$ is such that $\N \sm B$ is finite, define $[A,B] \sq X$ as 
        \[[A,B] = \{E \sq \N: A \sq E \sq B\}.\]
    Prove that the collection of $[A,B]$ form a basis for a topology on $X$. Prove that with respect to this topology, $X$ is Hausdorff and disconnected. Prove that the function $f: X \times X \to X$ given by 
        \[f(E_1,E_2) = E_1 \cap E_2\]
    is continuous.
\end{prob}

A collection of subsets of a set $X$ is a \emph{basis} if every open set in $X$ can be written as the union of a subfamily of subsets in the collection.\\

To check if a collection $\scr{B}$ forms a basis for $X$, it suffices to show that $\scr{B}$ covers $X$ and that given $B_1,B_2 \in \scr{B}$ and $x \in B_1 \cap B_2$, there exists $B_3 \in \scr{B}$ such that $x \in B_3 \sq B_1 \cap B_2$.\\

A collection $\scr{B}$ is a basis for a topological space $X$ if every set in $\scr{B}$ is open in $X$ and for any point $x \in X$ and open set $U$ containing $x$, there exists a set $B \in \scr{B}$ such that $x \in \scr{B} \sq U$.\\

\begin{proof}
    Let $\scr{B}$ denote the collection of all $[A,B]$ with $A \sq \N$ finite and $B \sq \N$ cofinite. Let $E \sq \N$ be an arbitrary element in $X$. Then, $E \in [\varnothing, \N] \in \scr{B}$. That is, the collection $\scr{B}$ covers $X$.\\

    Suppose now that $[A_1,B_1],[A_2,B_2] \in \scr{B}$.  If $E \in [A_1,B_1] \cap [A_2,B_2]$ then $A_1 \cup A_2 \sq E \sq B_1 \cap B_2$. But $A_1$ and $A_2$ being finite implies that $A_1 \cup A_2$ is finite. Similarly, since both $\N \sm B_1$ and $\N \sm B_2$ are finite, $\N \sm (B_1 \cap B_2)$ is finite. Therefore, $E \in [A_1\cup A_2, B_1 \cap B_2] \in \scr{B}$.
\end{proof}

\begin{proof}
    Let $E, F \sq \N$ be distinct subsets. Without loss of generality, there exists $n \in E \sm F$. Then, $E \in [\{n\}, \N]$, $F \not\in [\{n\}, \N]$, and $[\{n\}, \N] \in \scr{B}$. Also, $E \not\in [\varnothing, \N - \{n\}]$, $F \in [\varnothing, \N - \{n\}]$, and $[\varnothing, \N - \{n\}] \in \scr{B}$. Clearly $[\{n\}, \N]$ and $[\varnothing, \N - \{n\}]$ are disjoint. Thus, $X$ with respect to this topology is Hausdorff.\\
\end{proof}

\begin{proof}
    Notice that any set $G \sq \N$ either contains $n$ or does not contain $n$. This means that $G \in [\{n\}, \N]$ or $G \in [\varnothing, \N - \{n\}]$. Since $X$ can be written as the disjoint union of two nonempty open sets, $X$ is disconnected.
\end{proof}

\begin{proof}
    Let  $f: X \times X \to X$ be given by 
    \[f(E_1,E_2) = E_1 \cap E_2.\]
    To show that $f$ is continuous, we use the neighborhood definition of continuity: $f$ is continuous if given an arbitrary point $(E_1,E_2) \in X \times X$ and an open set $V$ containing $f(E_1,E_2)$, there exists an open set $U$ containing $(E_1,E_2)$ such that $f(U) \sq V$.\\
    
    Fix $(E_1,E_2) \in X \times X$ and let $[A,B]$ be an arbitrary basic open set in $X$ containing $E_1 \cap E_2$. Then $A \sq E_1 \cap E_2 \sq B$.\\

    Define $B_1 = B \cup (E_1 \sm E_2)$ and $B_2 = B \cup (E_2 \sm E_1)$. Then, $E_1 = (E_1 \cap E_2) \cup (E_1 \sm E_2) \sq B_1$ and similarly, $E_2 \sq B_2$. Since $\N \sm B$ is finite, it follows that both $\N \sm B_1$ and $\N \sm B_2$ are finite. Also, $A \sq E_1\cap E_2 \sq E_1$ and $A \sq E_1 \cap E_2 \sq E_2$. Therefore, $E_1 \in [A,B_1]$ and $E_2 \in [A,B_2]$. The set $[A,B_1] \times [A,B_2]$ is a basic open set in $X \times X$. Furthermore, for any $(F_1,F_2) \in [A,B_1] \times [A,B_2]$,
        \[A \sq F_1\cap F_2 \sq B_1 \cap B_2 = B.\]
    That is, $f\left([A,B_1] \times [A,B_2]\right) \sq [A,B]$.
\end{proof}


\begin{prob}{(S20.6)}{S20.6}
    Prove that the product of two regular spaces is regular.
\end{prob}

\begin{proof}
    See \ref{thm:prodRegular}.
\end{proof}

\begin{prob}{(F19.6)}{F19.6}
    Let $X$ be a compact topological space. Give a proof or counterexample for the following:

    \begin{enumerate}[(a)]
        \item Let $\{F_k\}$ be a decreasing, nested sequence of non-empty closed subsets of $X$. Then, $\bigcap_{k=1}^\infty F_k \neq \varnothing$.
        \item Let $\{O_k\}$ be a decreasing, nested sequence of non-empty open subsets of $X$. Then, $\bigcap_{k=1}^\infty O_k \neq \varnothing$.
    \end{enumerate}
\end{prob}

\begin{proof}
    This is true: see \ref{prob:nestedClosed}.
\end{proof}

\begin{solution}
    This is false: see \ref{prob:nestedOpen}.
\end{solution}


\begin{prob}{(F06.1)}{F06.1}
    Let $X$ and $Y$ be topological spaces.
    
    \begin{enumerate}[(a)]
        \item Define the product topology on $X \times Y$.
        \item Define what it means for a space $X$ to be connected.
        \item Show that $X$ and $Y$ are connected if and only if $X\times Y$ is connected.
    \end{enumerate}
\end{prob}

\begin{proof}
    See \ref{thm:productResults}.
\end{proof}

\begin{prob}{(F16.6), (S18.4)}{F16.6}
    Give an example of a space that is connected but not path-connected. Prove the example works.
\end{prob}


\begin{solution}
    Consider the topologist's sine curve. See \ref{thm:topSine} for the details.
\end{solution}

\begin{prob}{F16.2}{F16.2}
    Give a proof or counterexample for the following:

    \begin{enumerate}[(a)]
        \item Every closed subset of a compact space is compact.
        \item The product of any two connected spaces is connected.
    \end{enumerate}
\end{prob}

\begin{proof}
    See \ref{thm:compactClosed}.
\end{proof}

\begin{proof}
    See \ref{thm:productResults}.
\end{proof}

\begin{prob}{S17.2}{S17.2}
    Let $X$ be a compact space, $Y$ a topological space, and $\scr{C}$ an open cover of $X \times Y$. Prove that for all $y \in Y$ there exists an open neighborhood $U$ of $y$ such that $X \times U$ is contained in the union of finitely many elements from $\scr{C}$.
\end{prob}

\begin{proof}
    Fix $y \in Y$ and notice that $X \cong X \times \{y\}$. Therefore $X \times \{y\}$  is also compact and since $\scr{C}$ is an open cover for $X \times \{y\}$, there exists a finite subcover, say $\{W_1,\ldots, W_n\}$. Recall that every open set in $X \times Y$ can be written as a union of sets of the form $V_\alpha \times U_\alpha$ where $V_\alpha \sq X$ and $U_\alpha \sq Y$ are both open. Define $U$ to be the union of the $U_\alpha$ that generate the $W_k$. Then $U$ is a union of open sets in $Y$ that are open. Since $X \times \{y\} \sq \bigcup{k=1}^n W_k$, $y \in W_k$ for some $k$. Since $U$ was created from the basic open sets for $W_k$, $y \in U$. By construction of $U$ and choice in the cover, $X \times U \sq \bigcup_{k=1}^n W_k$, as desired.
\end{proof}


\begin{prob}{F05.1, F14.4}{F05.1}
    A space $X$ is step connected if given any open covering $\scr{U}$ of $X$ and any pair of points $p,q \in X$ there exists a finite sequence $U_1,\ldots,U_n$ of sets in $\scr{U}$ such that $p \in U_1$, $q \in U_n$ and $U_i \cap U_{i+1} \neq \varnothing$ for each $1 \leq i \leq n-1$. Prove that a space is step connected if and only if it is connected.
\end{prob}

\begin{proof}
    Assume that $X$ is step connected and suppose that $U,V$ are nonempty, disjoint, open sets such that $X = U \cup V$. Let $p \in U$ and $q\in V$. Since $\scr{U} = \{U,V\}$ is a collection of open sets there exists a finite sequence of sets in $\scr{U}$ connecting $p$ to $q$. Since $U \cap V = \varnothing$, it is impossible to form the step connection, a contradiction. Therefore $X$ is connected.\\

    Assume now that $X$ is connected and let $\scr{U} = \{U_\alpha\}$ be a collection of open sets. Let $p,q \in X$ be arbitary. Construct a sequence of open sets as follows: let $V_0$ be any $U_\alpha \in \scr{U}$ and let $V_1$ be the union of each $U_\alpha \in \scr{U}$ that has nonempty intersection with $V_0$. For each $n \in \N$, inductively define $V_n$ to be the union of all $U_\alpha$ in $\scr{U}$ that have nonempty intersection with $V_{n-1}$. By construction, each $V_n$ is an open set and therefore $V = \bigcup_{n=1}^\infty V_n$ is also open.\\
    
    Seeking a contraction, suppose that $q \not\in V$. Notice that $X - V$ is the union of the $U_\alpha$ that are disjoint from $V$ and therefore $X - V$ is open. But this implies that $V$ is both open and closed. Since $X$ is connected, either $V = X$ or $V = \varnothing$. Both of these are impossible since $q \not\in V$ and $p \in V$.
\end{proof}

\begin{prob}{S17.1}{S17.1}
    \begin{enumerate}[(a)]
        \item Any quotient of a Hausdorff space is Hausdorff.
        \item Any metric space is normal.
        \item If $X$ is a topological space and $A \sq B \sq X$ and $\ol{A}$ is the closure of $A$ in $X$, then $\ol{A} \cap B$ is the closure of $A$ with respect to the subspace topology on $B$.
    \end{enumerate}
\end{prob}

\begin{solution}
    This is false. Consider $X = [0,2]$ and $A = (1,2]$ where $X$ is equipped with the usual topology. Then $X$ is Hausdorff, but $X/A$ is not Hausdorff since 1 cannot be separated from $A$.
\end{solution}

\begin{proof}
    This is true: see \ref{thm:metNormal}.
\end{proof}

\begin{proof}
    Let $C$ denote the closure of $A$ in $B$.  Since $\ol{A}$ is closed in $X$, $\ol{A} \cap B$ is a closed set in $B$ with respect to the subspace topology. Since $A \sq B$ and $A \sq \ol{A}$, $A \sq \ol{A} \cap B$. But, $C$ is the smallest closed set in $B$ that contains $A$ and thus $C \sq \ol{A} \cap B$.\\

    On the other hand, $C$ is closed in $B$. Then $C = C' \cap B$ for some set $C' \sq X$ that is closed in $X$. Since $A \sq C$ by definition of closure, $A \sq C'$. But, $\ol{A}$ is the smallest closed set containing $A$ and therefore $\ol{A} \sq C'$. Therefore, $\ol{A}\cap B \sq C' \cap B = C$.
\end{proof}


\begin{prob}{F13}{F13.3}
    Prove or provide a counter example to the following:
    \begin{enumerate}[(a)]
        \item The interior of a connected set is connected.
        \item The closure of a path connected set is path connected.
        \item The quotient of a connected set is connected (under the quotient topology).
        \item If $C$ is an infinite collection of connected sets where every pair of sets in $C$ has a non-empty intersection then its union is connected.
    \end{enumerate}
\end{prob}

\begin{solution}
The interior of a connected set need not be connected. Let $X \sq \R^2$ be the closed unit ball with center $(0,1)$ and $Y \sq \R^2$ the closed unit ball with center (0,-1). Then $X \cup Y$ is connected as the set is path-connected. However, the interior of $X \cup Y$ is the union of the corresponding open balls. In this case, the open balls provide a separation meaning that the interior is not connected.
\end{solution}

\begin{solution}
    The closure of a path connected set need not be path connected. Consider the Topologist's Spiral. Let $X$ denote the spiral and $Y = S^1$ so that the Topologist's Spiral can be written as $X \cup Y$. In this case, $X$ is path-connected, but the closure of $X$ in $X \cup Y$ is $X \cup Y$ which is not path-connected.
\end{solution}

\begin{proof}
    Let $X$ be a connected set and $\sim$ some equivalence relation on $X$. Let $Y = X/\sim$. The quotient map $q: X \to X/\sim$ is a surjective, continuous map. As the continuous image of a connected set is connected, it follows that $X/\sim$ is connected.
\end{proof}

\begin{proof}
    \textcolor{red}{Help!}
\end{proof}

\begin{prob}{F12}{F12.1}
    Suppose $X,Y$ are topological spaces and $A \sq X$ and $B\sq Y$. Prove that
    \begin{enumerate}[(a)]
        \item $\Int{A \times B} = \Int{A} \times \Int{B}$.
        \item $\cl{A \times B} = \cl{A} \times \cl{B}$.
        \item $\partial(A \times B) = [\partial(A) \times \cl{B}]\cup [\cl{A} \times \partial(B)]$.
    \end{enumerate}
\end{prob}

\begin{proof}
    Let $(x,y) \in \Int{A \times B}$. There exists a basic open set $U \times V \sq A \times B$ such that $(x,y) \in U \times V$. Then $U \sq A$ is open in $X$ and $x \in U$ meaning that $x \in \Int{A}$. Similarly, $V \sq B$ is open in $Y$ and $y \in V$ and therefore $y \in \Int{B}$. This means that $(x,y) \in \Int{A} \times \Int{B}$.\\

    Conversely, suppose that $(x,y) \in \Int{A} \times \Int{B}$. Choose open sets $U \sq A$ and $V \sq B$ that contain $x$ and $y$, respectively. Then, $U \times V$ is a basic open set in $X \times Y$ that contains $(x,y)$ and is contained in $A \times B$. Thus $(x,y) \in \Int{A \times B}$.
\end{proof}

\begin{proof}
    Suppose that $(x,y) \in \cl{A\times B}$. If $(x,y) \in A \times B$ then $(x,y) \in \cl{A} \times \cl{B}$ as the closure of any set must contain the original set. Suppose now that $(x,y)$ is a boundary point of $A \times B$. Let $U \times V$ be a basic open set about $(x,y)$. Since $(x,y)$ is a boundary point of $A \times B$, $(A \times B) \cap (U \times V) \neq \varnothing$ and $(X - A \times Y - B) \cap (U \times V) \neq \varnothing$. In particular, $A \cap U$ and $X - A \cap U$ are both nonempty meaning that $x$ is a boundary point of $A$. Similarly, $y$ is a boundary point of $B$. Therefore, $(x,y) \in \cl{A} \times \cl{B}$.\\

    Conversely, suppose that $(x,y) \in \cl{A} \times \cl{B}$. If $x \in A$ and $y \in B$, then $(x,y) \in A \times B$.\\
    
    Suppose that $x$ is a boundary point of $A$ and $y \in B$. Let $U \times V$ be a basic open set in $X \times Y$ that contains $(x,y)$. Then $U$ is an open set in $X$ that contains $x$. Since $x$ is a boundary point of $A$, both $(X - A) \cap U$ and $A \cap U$ are nonempty. By assumption, $B \cap V$ is nonempty as it contains $y$. Therefore,
        \[(A \times B) \cap (U \times V) = (A \cap U) \times (B\cap V) \neq \varnothing.\]
    Observe that
        \[\big( (X\times Y) - (A \times B)\big) \cap (U \times V) = \big((X - A) \times Y\big) \cup \big(X\times (Y - B)\big) \cap (U \times V) \]
    and since $\big((X - A) \times Y\big) \cap (U \times V) \neq \varnothing$, $\big( (X\times Y) - (A \times B)\big) \cap (U \times V) \neq\varnothing$. That is, $(x,y)$ is a boundary point of $A \times B$ and therefore $(x,y) \in \cl{A\times B}$.\\

    An identical proof shows that $(x,y) \in \cl{A\times B}$ if $x \in A$ and $y$ is a boundary point of $B$. If both $x$ and $y$ are boundary points of $A$ and $B$ respectively, then $(x,y) \in \cl{A\times B}$ since it is a boundary point of $A \times B$.
\end{proof}

\emph{The proof for (c) follows from my proof for (b). Is there a better way for me to have proved (b)?}

\begin{prob}{F20}{F20.1}
\begin{enumerate}[(a)]
\item Give an example of two topological spaces $X,Y$ and a continuous bijection $f: X \to Y$ that is not a homeomorphism.
\item Show that if $X$ is compact and $Y$ is Hausdorff, then every continuous bijection between the spaces is a homeomorphism.
\end{enumerate}
\end{prob}

\begin{solution}
Let $X = [0,1]$ with the standard topology and $Y = [0,1]$ with the trivial topology. Let $f: X \to Y$ be the identity map. Clearly $f$ is bijective. The only open sets in $Y$ are $\varnothing $ and $[0,1]$. Since both $f\inv(\varnothing) = \varnothing$ and $f\inv([0,1]) = [0,1]$ are open in $X$, $f$ is continuous. However, $f$ is not a homeomorphism since $(0,1)$ is open in $X$ but $f(0,1) = (0,1)$ is not open in $Y$.
\end{solution}

\begin{proof}
Let $f: X \to Y$ be a continuous bijection from a compact space to a Hausdorff space. To show that $f$ is a homeomorphism, it remains to check that $f$ is an open mapping. This is equivalent to proving that $f$ maps closed sets to closed sets. Let $A \sq X$ be a closed set. Since $X$ is compact, $A$ is compact in $X$. Then, $f(A) \sq Y$ must be compact since $f$ is continuous. In a Hausdorff space, any compact set is closed and thus $f(A)$ is closed in $Y$, as desired.
\end{proof}

\begin{prob}{S12.3, F11.6}{S12.3}
    Prove the following:
    \begin{enumerate}[(a)]
        \item A closed subspace of a compact space is compact.
        \item A compact subspace of a Hausdorff space is closed.
        \item If $f: X \to Y$ is a continuous bijection, $X$ is compact and $Y$ is Hausdorff, then $f$ is a homeomorphism.
    \end{enumerate}
\end{prob}

\begin{proof}
    Suppose that $A \sq X$ is a closed subspace of a compact space. Let $\{U_i\}_{i \in I}$ be an open cover of $A$. Extend this collection to an open cover of $X$ by appending the open set $X - A$. Because $X$ is compact, there exists a finite subcover of $X$, say $\{U_1, \ldots, U_n\}$. If some $U_j = X - A$, remove this $U_j$ from the list to obtain a finite subcover for $A$, from the original collection of open sets. As any open cover of $A$ has a finite subcover, $A$ is compact.
\end{proof}

\begin{proof}
    Assume that $A \sq X$ is a compact subspace of a Hausdorff space. To prove that $A$ is closed, we prove that $X - A$ is open. Let $x \in X - A$. Because $X$ is Hausdorff, for each $a \in A$ there exist open neighborhoods $U_a$ of $x$ and $V_a$ of $a$ where $U_a \cap V_a = \varnothing$. Then, the collection $\{V_a\}_{a \in A}$ forms an open cover of $A$. Since $A$ is compact, there exists a finite subcover, say $\{V_{a_1}, \ldots, V_{a_n}\}$. Then, $U = \bigcap_{i=1}^n U_{a_i}$ is an open set containing $x$ that is disjoint from $A$ and thus is contained in $X - A$. Therefore, $X - A$ is open and so $A$ is closed.
\end{proof}

\begin{proof}
    See \ref{prob:F20.1}.
\end{proof}

\begin{prob}{W08.1, S12.2}{S12.2}
    Let $X,Y,T$ be topological spaces.

    \begin{enumerate}[(a)]
        \item Define the product topology on $X \times Y$.
        \item Show that the projection functions $p_X: X \times Y \to X$ and $p_Y: X \times Y \to Y$ are continuous.
        \item Show that a function $f: T \to X \times Y$ is continuous if and only if both $p_X \circ f$ and $p_Y \circ f$ are continuous.
        \item Show that the product topology on $X \times Y$ is the unique topology that for all spaces $T$ and functions $f$, (c) is satisfied.
    \end{enumerate}
\end{prob}

Let $X,Y$ be topological spaces. The product topology on $X \times Y$ has a basis given by $U \times V$ where $U \sq X$ is open and $V \sq Y$ is open. That is, any open set in $X \times Y$ with respect to the product topoology is the union of sets of the form $U \times V$.

\begin{proof}
Let $p_X: X\times Y \to X$ be the projection function onto $X$. Let $U \sq X$ be an open set. Then,
    \[p_X\inv(U) = U \times Y.\]
Because $U$ is open in $X$ and $Y$ is open in $Y$, $U \times Y$ is open in $X \times Y$. Therefore $p_X$ is continuous. Similarly, for any open subset $V$ of $Y$,
    \[p_Y\inv(V) = X \times V\]
which is open in $X \times Y$. Whence both projection functions are continuous.
\end{proof}

\begin{proof}
    Assume that $f: T \to X \times Y$ is continuous. Let $U \sq X$ and $V \sq Y$ be arbitrary open subsets. Because $p_X$ is continuous, $p_X\inv(U)$ is open in $X \times Y$. Since $f$ is continuous, $f\inv(p_X\inv(U))$ is open in $T$. Therefore, $(p_X \circ f)\inv(U)$ is open in $T$ implying that $p_X \circ f$ is continuous. Similarly, $p_Y\inv(V)$ is open in $X \times Y$ and therefore $f\inv(p_Y\inv(V))$ is open in $T$. This implies that $p_Y \circ f$ is continuous.\\ 

    Now assume that both $p_X \circ f$ and $p_Y \circ f$ are continuous. Let $U \times V$ be an arbitary basic open set in $X \times Y$. Then $U \sq X$ and $V \sq Y$ are both open. Because the projections are continuous, both $p_X\inv(U)$ and $p_Y\inv(V)$ are open in $X \times Y$. Let $t \in f\inv(U \times V)$. If $f(t) = (x,y)$ then $x \in U$ and $y \in V$. This means that $p_X(f(t)) = x \in U$ and $p_Y(f(t)) = y \in V$. That is, $t \in f\inv(p_X\inv(U)) \cap f\inv(p_Y\inv(V))$. Note that the reverse of each of these implications holds and therefore $f\inv(U \times V) = f\inv(p_X\inv(U)) \cap f\inv(p_Y\inv(V))$. As $U$ and $V$ are open and the the compositions are assumed to be continuous, $f\inv(U \times V)$ is the intersection of two open sets and thus must also be open. Since $U \times V$ was an arbitrary basic open set, $f$ is continuous.
\end{proof}

\begin{proof}
    Let $T = X \times Y$ under an arbitrary topology. The identity map $\ident: T \to T$ is continuous and therefore both $p_X\circ \ident: T \to X$ and $p_Y \circ \ident: T \to Y$ are continuous. That is, for any open sets $U \sq X$ and $V \sq Y$,
        \[(p_X \circ \ident)\inv(U) = U \times Y \]
    and
        \[(p_Y \circ \ident)\inv(V) = X \times V \]
    are both open in $T$. As a finite intersection of open sets is open, $(U \times Y) \cap (X \times V) = U \times V$ is open in $T$ whenever $U$ is open in $X$ and $V$ is open in $Y$. That is, every basis element for the product topology is open in $T$ as well.\\

    \textcolor{red}{Worried about reverse direction here.}
    
    Now consider the identity map $\ident: T \to X \times Y$. Let $U \times V \sq X \times Y$ be a basic open set for the product topology. Then,
        \[(p_X \circ \ident)\inv( U \times V) = \ident\inv(U \times Y) = U \times Y\]
    and
        \[(p_Y \circ \ident)\inv( U \times V) = \ident\inv(X \times V) = X \times V.\]
    Since both $U \times Y$ and $X \times V$ are open in $X \times Y$, 
\end{proof}

\end{document}