\documentclass[../TopologyQualSolutions.tex]{subfiles}

\begin{document}

\section{Basic Point Set Topology}
\begin{prob}{F13}{F13.3}
    Prove or provide a counter example to the following:
    \begin{enumerate}[(a)]
        \item The interior of a connected set is connected.
        \item The closure of a path connected set is path connected.
        \item The quotient of a connected set is connected (under the quotient topology).
        \item If $C$ is an infinite collection of connected sets where every pair of sets in $C$ has a non-empty intersection then its union is connected.
    \end{enumerate}
\end{prob}

\begin{solution}
The interior of a connected set need not be connected. Let $X \sq \R^2$ be the closed unit ball with center $(0,1)$ and $Y \sq \R^2$ the closed unit ball with center (0,-1). Then $X \cup Y$ is connected as the set is path-connected. However, the interior of $X \cup Y$ is the union of the corresponding open balls. In this case, the open balls provide a separation meaning that the interior is not connected.
\end{solution}

\begin{solution}
    The closure of a path connected set need not be path connected. Consider the Topologist's Spiral. Let $X$ denote the spiral and $Y = S^1$ so that the Topologist's Spiral can be written as $X \cup Y$. In this case, $X$ is path-connected, but the closure of $X$ in $X \cup Y$ is $X \cup Y$ which is not path-connected.
\end{solution}

\begin{proof}
    Let $X$ be a connected set and $\sim$ some equivalence relation on $X$. Let $Y = X/\sim$. The quotient map $q: X \to X/\sim$ is a surjective, continuous map. As the continuous image of a connected set is connected, it follows that $X/\sim$ is connected.
\end{proof}

\begin{proof}
    \textcolor{red}{Help!}
\end{proof}

\begin{prob}{F12}{F12.1}
    Suppose $X,Y$ are topological spaces and $A \sq X$ and $B\sq Y$. Prove that
    \begin{enumerate}[(a)]
        \item $\Int{A \times B} = \Int{A} \times \Int{B}$.
        \item $\cl{A \times B} = \cl{A} \times \cl{B}$.
        \item $\partial(A \times B) = [\partial(A) \times \cl{B}]\cup [\cl{A} \times \partial(B)]$.
    \end{enumerate}
\end{prob}

\begin{proof}
    Let $(x,y) \in \Int{A \times B}$. There exists a basic open set $U \times V \sq A \times B$ such that $(x,y) \in U \times V$. Then $U \sq A$ is open in $X$ and $x \in U$ meaning that $x \in \Int{A}$. Similarly, $V \sq B$ is open in $Y$ and $y \in V$ and therefore $y \in \Int{B}$. This means that $(x,y) \in \Int{A} \times \Int{B}$.\\

    Conversely, suppose that $(x,y) \in \Int{A} \times \Int{B}$. Choose open sets $U \sq A$ and $V \sq B$ that contain $x$ and $y$, respectively. Then, $U \times V$ is a basic open set in $X \times Y$ that contains $(x,y)$ and is contained in $A \times B$. Thus $(x,y) \in \Int{A \times B}$.
\end{proof}

\begin{proof}
    Suppose that $(x,y) \in \cl{A\times B}$. If $(x,y) \in A \times B$ then $(x,y) \in \cl{A} \times \cl{B}$ as the closure of any set must contain the original set. Suppose now that $(x,y)$ is a boundary point of $A \times B$. Let $U \times V$ be a basic open set about $(x,y)$. Since $(x,y)$ is a boundary point of $A \times B$, $(A \times B) \cap (U \times V) \neq \varnothing$ and $(X - A \times Y - B) \cap (U \times V) \neq \varnothing$. In particular, $A \cap U$ and $X - A \cap U$ are both nonempty meaning that $x$ is a boundary point of $A$. Similarly, $y$ is a boundary point of $B$. Therefore, $(x,y) \in \cl{A} \times \cl{B}$.\\

    Conversely, suppose that $(x,y) \in \cl{A} \times \cl{B}$. If $x \in A$ and $y \in B$, then $(x,y) \in A \times B$.\\
    
    Suppose that $x$ is a boundary point of $A$ and $y \in B$. Let $U \times V$ be a basic open set in $X \times Y$ that contains $(x,y)$. Then $U$ is an open set in $X$ that contains $x$. Since $x$ is a boundary point of $A$, both $(X - A) \cap U$ and $A \cap U$ are nonempty. By assumption, $B \cap V$ is nonempty as it contains $y$. Therefore,
        \[(A \times B) \cap (U \times V) = (A \cap U) \times (B\cap V) \neq \varnothing.\]
    Observe that
        \[\big( (X\times Y) - (A \times B)\big) \cap (U \times V) = \big((X - A) \times Y\big) \cup \big(X\times (Y - B)\big) \cap (U \times V) \]
    and since $\big((X - A) \times Y\big) \cap (U \times V) \neq \varnothing$, $\big( (X\times Y) - (A \times B)\big) \cap (U \times V) \neq\varnothing$. That is, $(x,y)$ is a boundary point of $A \times B$ and therefore $(x,y) \in \cl{A\times B}$.\\

    An identical proof shows that $(x,y) \in \cl{A\times B}$ if $x \in A$ and $y$ is a boundary point of $B$. If both $x$ and $y$ are boundary points of $A$ and $B$ respectively, then $(x,y) \in \cl{A\times B}$ since it is a boundary point of $A \times B$.
\end{proof}

\emph{The proof for (c) follows from my proof for (b). Is there a better way for me to have proved (b)?}

\begin{prob}{F20}{F20.1}
\begin{enumerate}[(a)]
\item Give an example of two topological spaces $X,Y$ and a continuous bijection $f: X \to Y$ that is not a homeomorphism.
\item Show that if $X$ is compact and $Y$ is Hausdorff, then every continuous bijection between the spaces is a homeomorphism.
\end{enumerate}
\end{prob}

\begin{solution}
Let $X = [0,1]$ with the standard topology and $Y = [0,1]$ with the trivial topology. Let $f: X \to Y$ be the identity map. Clearly $f$ is bijective. The only open sets in $Y$ are $\varnothing $ and $[0,1]$. Since both $f\inv(\varnothing) = \varnothing$ and $f\inv([0,1]) = [0,1]$ are open in $X$, $f$ is continuous. However, $f$ is not a homeomorphism since $(0,1)$ is open in $X$ but $f(0,1) = (0,1)$ is not open in $Y$.
\end{solution}

\begin{proof}
Let $f: X \to Y$ be a continuous bijection from a compact space to a Hausdorff space. To show that $f$ is a homeomorphism, it remains to check that $f$ is an open mapping. This is equivalent to proving that $f$ maps closed sets to closed sets. Let $A \sq X$ be a closed set. Since $X$ is compact, $A$ is compact in $X$. Then, $f(A) \sq Y$ must be compact since $f$ is continuous. In a Hausdorff space, any compact set is closed and thus $f(A)$ is closed in $Y$, as desired.
\end{proof}

\begin{prob}{S12}{S12.3}
    Prove the following:
    \begin{enumerate}[(a)]
        \item A closed subspace of a compact space is compact.
        \item A compact subspace of a Hausdorff space is closed.
        \item If $f: X \to Y$ is a continuous bijection, $X$ is compact and $Y$ is Hausdorff, then $f$ is a homeomorphism.
    \end{enumerate}
\end{prob}

\begin{proof}
    Suppose that $A \sq X$ is a closed subspace of a compact space. Let $\{U_i\}_{i \in I}$ be an open cover of $A$. Extend this collection to an open cover of $X$ by appending the open set $X - A$. Because $X$ is compact, there exists a finite subcover of $X$, say $\{U_1, \ldots, U_n\}$. If some $U_j = X - A$, remove this $U_j$ from the list to obtain a finite subcover for $A$, from the original collection of open sets. As any open cover of $A$ has a finite subcover, $A$ is compact.
\end{proof}

\begin{proof}
    Assume that $A \sq X$ is a compact subspace of a Hausdorff space. To prove that $A$ is closed, we prove that $X - A$ is open. Let $x \in X - A$. Because $X$ is Hausdorff, for each $a \in A$ there exist open neighborhoods $U_a$ of $x$ and $V_a$ of $a$ where $U_a \cap V_a = \varnothing$. Then, the collection $\{V_a\}_{a \in A}$ forms an open cover of $A$. Since $A$ is compact, there exists a finite subcover, say $\{V_{a_1}, \ldots, V_{a_n}\}$. Then, $U = \bigcap_{i=1}^n U_{a_i}$ is an open set containing $x$ that is disjoint from $A$ and thus is contained in $X - A$. Therefore, $X - A$ is open and so $A$ is closed.
\end{proof}

\begin{proof}
    See \ref{prob:F20.1}.
\end{proof}

\begin{prob}{S12}{S12.2}
    Let $X,Y,T$ be topological spaces.

    \begin{enumerate}[(a)]
        \item Define the product topology on $X \times Y$.
        \item Show that the projection functions $p_X: X \times Y \to X$ and $p_Y: X \times Y \to Y$ are continuous.
        \item Show that a function $f: T \to X \times Y$ is continuous if and only if both $p_X \circ f$ and $p_Y \circ f$ are continuous.
        \item Show that the product topology on $X \times Y$ is the unique topology that for all spaces $T$ and functions $f$, (c) is satisfied.
    \end{enumerate}
\end{prob}

Let $X,Y$ be topological spaces. The product topology on $X \times Y$ has a basis given by $U \times V$ where $U \sq X$ is open and $V \sq Y$ is open. That is, any open set in $X \times Y$ with respect to the product topoology is the union of sets of the form $U \times V$.

\begin{proof}
Let $p_X: X\times Y \to X$ be the projection function onto $X$. Let $U \sq X$ be an open set. Then,
    \[p_X\inv(U) = U \times Y.\]
Because $U$ is open in $X$ and $Y$ is open in $Y$, $U \times Y$ is open in $X \times Y$. Therefore $p_X$ is continuous. Similarly, for any open subset $V$ of $Y$,
    \[p_Y\inv(V) = X \times V\]
which is open in $X \times Y$. Whence both projection functions are continuous.
\end{proof}

\begin{proof}
    Assume that $f: T \to X \times Y$ is continuous. Let $U \sq X$ and $V \sq Y$ be arbitrary open subsets. Because $p_X$ is continuous, $p_X\inv(U)$ is open in $X \times Y$. Since $f$ is continuous, $f\inv(p_X\inv(U))$ is open in $T$. Therefore, $(p_X \circ f)\inv(U)$ is open in $T$ implying that $p_X \circ f$ is continuous. Similarly, $p_Y\inv(V)$ is open in $X \times Y$ and therefore $f\inv(p_Y\inv(V))$ is open in $T$. This implies that $p_Y \circ f$ is continuous.\\ 

    Now assume that both $p_X \circ f$ and $p_Y \circ f$ are continuous. Let $U \times V$ be an arbitary basic open set in $X \times Y$. Then $U \sq X$ and $V \sq Y$ are both open. Because the projections are continuous, both $p_X\inv(U)$ and $p_Y\inv(V)$ are open in $X \times Y$. Let $t \in f\inv(U \times V)$. If $f(t) = (x,y)$ then $x \in U$ and $y \in V$. This means that $p_X(f(t)) = x \in U$ and $p_Y(f(t)) = y \in V$. That is, $t \in f\inv(p_X\inv(U)) \cap f\inv(p_Y\inv(V))$. Note that the reverse of each of these implications holds and therefore $f\inv(U \times V) = f\inv(p_X\inv(U)) \cap f\inv(p_Y\inv(V))$. As $U$ and $V$ are open and the the compositions are assumed to be continuous, $f\inv(U \times V)$ is the intersection of two open sets and thus must also be open. Since $U \times V$ was an arbitrary basic open set, $f$ is continuous.
\end{proof}

\begin{proof}
    Let $T = X \times Y$ under an arbitrary topology. The identity map $\ident: T \to T$ is continuous and therefore both $p_X\circ \ident: T \to X$ and $p_Y \circ \ident: T \to Y$ are continuous. That is, for any open sets $U \sq X$ and $V \sq Y$,
        \[(p_X \circ \ident)\inv(U) = U \times Y \]
    and
        \[(p_Y \circ \ident)\inv(V) = X \times V \]
    are both open in $T$. As a finite intersection of open sets is open, $(U \times Y) \cap (X \times V) = U \times V$ is open in $T$ whenever $U$ is open in $X$ and $V$ is open in $Y$. That is, every basis element for the product topology is open in $T$ as well.\\

    \textcolor{red}{Worried about reverse direction here.}
    
    Now consider the identity map $\ident: T \to X \times Y$. Let $U \times V \sq X \times Y$ be a basic open set for the product topology. Then,
        \[(p_X \circ \ident)\inv( U \times V) = \ident\inv(U \times Y) = U \times Y\]
    and
        \[(p_Y \circ \ident)\inv( U \times V) = \ident\inv(X \times V) = X \times V.\]
    Since both $U \times Y$ and $X \times V$ are open in $X \times Y$, 
\end{proof}

\end{document}