\documentclass[../TopologyQualSolutions.tex]{subfiles}

\begin{document}

\section{Metric Spaces}

\begin{prob}{F13}{F13.5}
    Let $X$ be a complete metric space and $\{C_n\}_{n \in \N}$ a collection of non-empty closed sets such that $C_1 \supseteq C_2 \supseteq \cdots$. Assume that the sequence of diameters of the $C_n$ goes to zero. Prove that the intersection $\cap C_n$ of this collection is nonempty.
\end{prob}

\begin{proof}
    Construct a sequence $\{x_n\}$ by choosing any $x_i \in C_i$ for each $i = 1,2,\ldots$. Because the sets are nested, $x_n \in C_k$ whenever $k \leq n$ for each $n \in \N$.\\

    Let $\{r_n\}$ be the sequence of diameters of the $C_n$. By assumption, $r_n \to 0$. Let $\epsilon > 0$ be arbitrary and choose $N \in \N$ where $n \geq N$ implies that $|r_n| < \epsilon$. Assume that $m,n \geq N$ and that $m \geq n$. Then,
        \[\norm[x_n - x_m] \leq r_n < \epsilon \]
    since $x_n,x_m \in C_n$. This means that $\{x_n\}$ is a Cauchy sequence in a complete space -- let $x \in X$ be the limit of $\{x_n\}$.\\

    To see that $x \in C_N$ for each $N$, notice that $\{x_n\}_{n \geq N}$ is a subseqeunce of $\{x_n\}$ that is contained in $C_N$. Since $x_n \to x$, this subseqeunce also converges to $x$ meaning that $x$ is a limit point of $C_N$. But, $C_N$ is closed and therefore contains all its limit points. Since $x \in \bigcap_{n=1}^\infty C_n$, the intersection is nonempty.
\end{proof}

\begin{prob}{S12}{S12.4}
    Suppose that $(X,d)$ is a metric space and $A \sq X$.
    \begin{enumerate}[(a)]
        \item For a fixed $x \in X$, define what is meant by $d(x,A)$.
        \item Show that for all $x,y \in X$, $d(x,A) \leq d(x,y) + d(y,A)$.
        \item Show that the function $f: X \to \R$ given by $f(x) = d(x,A)$ is a continuous function.
    \end{enumerate}
\end{prob}

Fix $x \in X$. Then $d(x,A) = \inf_{a \in A} d(x,a)$ describes the distance from $x$ to the set $A$.

\begin{proof}
    Let $x,y \in X$ be arbitrary. Because $d$ is a metric, for each $a \in A$, $d(x,a) \leq d(x,y) + d(y,a)$. Therefore,
        \[d(x,A) \leq d(x,a) \leq d(x,y) + d(y,a). \]
    This means that for each $a \in A$, $d(x,A) - d(x,y) \leq d(y,a)$. Because $d(y,A)$ is the infimum over all $d(y,a)$ with $a \in A$, it is the greatest lower bound. It then follows that $d(x,A) - d(x,y) \leq d(y,A)$, as desired.
\end{proof}

\begin{prob}{S20}{}
Let $(X,d)$ be a metric space and fix a point $x_0 \in X$. Let $\rho$ be a new metric given by $\rho(x,y) = d(x,x_0) + d(y,x_0)$ whenever $x \neq y$ and $\rho(x,y) = 0$ if $x = y$. Verify that $\rho$ is a metric and $(X,\rho)$ is complete.
\end{prob}

\begin{proof}
By construction, $\rho(x,y) \geq 0$ for each $x,y \in X$. Suppose $\rho(x,y) = 0$ but $x \neq y$. Then, $0 = \rho(x,y) = d(x,x_0) + d(y,y_0)$. Since at most one of $x$ and $y$ can be $x_0$, $d(x,x_0) + d(y,y_0) > 0$. Therefore $\rho(x,y) = 0$ if and only if $x = y$. Suppose now that $x,y, z \in X$. Then,
    \[\rho(x,y) + \rho(y,z) = d(x,x_0) + d(y,x_0) + d(y,x_0) + d(z,x_0) = \rho(x,z) + 2 d(y,x_0) \geq \rho(x,z) \]
proving that $\rho$ is a metric.\\

To see that $(X, \rho)$ is a complete metric space, let $(x_n)$ be a Cauchy sequence in $(X, \rho)$. Let $\epsilon > 0$ and choose $N \in \N$ sufficiently large such that $\rho(x_N, x_n) < \epsilon$ whenever $n \geq N$. This means that whenever $n \geq N$,
    \[ d(x_n,x_0) \leq d(x_N, x_0) + d(x_n, x_0) = \rho(x_N, x_m) < \epsilon. \]
Therefore, $x_n \to x_0$ in $(X,d)$. Equivalently, as $n \to \infty$, $d(x_n,x_0) \to 0$. Then,
    \[\rho(x_n, x_0) = d(x_n, x_0) + d(x_0, x_0) = d(x_n, x_0) \]
meaning that as $n \to \infty$, $\rho(x_n,x_0) \to 0$. That is, $x_n \to x_0$ in $(X,\rho)$.
\end{proof}



\end{document}