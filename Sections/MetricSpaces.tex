\documentclass[../TopologyQualSolutions.tex]{subfiles}

\begin{document}

\section{Metric Spaces}

\begin{prob}{(S19.3)}{S19.3}
    Let $X$ be a metric space with metric $d$.

    \begin{enumerate}[(a)]
        \item Suppose that there exists $\epsilon > 0$ such that for every $x \in X$, the open ball $B_\epsilon(x)$ has compact closure. Prove that $X$ is complete.
        \item Suppose that $X$ has the property that for every $x \in X$ there exists $\epsilon_x > 0$ such that $B_{\epsilon_x}(x)$ has compact closure. Give an example which shows $X$ need not be complete.
    \end{enumerate}
\end{prob}

\begin{proof}
    Let $\{x_n\}$ be a Cauchy sequence in $X$. Choose $N \in \N$ such that whenever $m,n \geq N$,
        \[d(x_m,x_n) < \epsilon. \]
    In particular, this means that whenever $n \geq N$, $x_n \in B_\epsilon(x_N)$. Therefore, $x_N, x_{N+1},\ldots$ is a Cauchy sequence contained in the compact space $\ol{B_\epsilon(x_N)}$. In a metric space, any compact space is sequentially compact (see \ref{thm:seqCompactMS}). Therefore, $\{x_n\}_{n=N}^\infty$ has a convergent subsequence that converges to some point $z \in \ol{B_\epsilon(x_N)}$.\\

    The convergent subsequence of $\{x_n\}_{n=N}^\infty$  is also a convergent subsequence of the original sequence. Since $\{x_n\}$ is Cauchy and has a subsequence converging to $z$, it follows that $x_n \to z$.
\end{proof}

\begin{solution}
    Consider the space $X = (0,1) \sq \R$. For any $x \in X$ there exists $\delta > 0$ such that $(x-\delta, x + \delta) \sq X$. Then, $\ol{B_{\delta/2}(x)} = [x - \delta/2,x + \delta/2]$ is compact. However, $X$ is not complete since $\frac{1}{n} \to 0 \not\in X$.
\end{solution}

\begin{prob}{(S20.5)}{S20.5}
    Let $(X,d)$ be a metric space and fix a point $x_0 \in X$. Let $\rho$ be a new metric defined by 
        \[\rho(x,y) = \begin{cases} 
            0 & x = y\\
            d(x,x_0) + d(y,x_0) & x \neq y
        \end{cases}\]
    Prove that $\rho$ is indeed a metric and that $(X,\rho)$ is complete.
\end{prob}

\begin{proof}
    By construction, $d(x,y) \geq 0$ for any $x, y \in X$. For any $x \in  X$, $d(x,x) = 0$. Assume now that $d(x,y) = 0$. If $x \neq y$, then at most one of $x$ and $y$ can be $x_0$. Therefore, $\rho(x,y) = d(x,x_0) + d(y,x_0) > 0$ when $x \neq y$. This means that $d(x,y) = 0$ if and only if $x = y$. Since $d(x,y) = d(y,x)$ for any $x,y \in X$, it follows that $\rho(x,y) = \rho(y,x)$ for any $x, y \in X$.\\

    Now let $x,y,z \in X$ be arbitrary. If $x = z$, then $\rho(x,z) = 0 \leq \rho(x,y) + \rho(y,z)$. If $x \neq z$,
        \[\rho(x,z) = d(x,x_0) + d(z,x_0) \leq d(x, x_0) + d(y, x_0) + d(y,x_0) + d(z,x_0) = \rho(x,y) + \rho(y,z).\]
\end{proof}

\begin{proof}
    Let $\{x_n\}$ be a Cauchy sequence in $(X,\rho)$ and $\epsilon > 0$. For any $m,n \in \N$,
        \[\rho(x_n,x_m) = d(x_n,x_0) + d(x_m,x_0).\]
    Since $\{x_n\}$ is Cauchy, for sufficiently large $m,n \in \N$,
        \[d(x_n,x_0) + d(x_m,x_0) = \rho(x_n,x_m) < \epsilon.\]
    Since both quantities on the left side of the equation are positive, this means that for sufficiently large $n \in \N$,
        \[d(x_n,x_0) < \epsilon. \]
    Then for sufficiently large $n$,
        \[\rho(x_n,x_0) = d(x_n,x_0) + d(x_0,x_0) = d(x_n,x_0) < \epsilon,\]
    proving that $\{x_n\}$ converges to $x_0$ in $(X,\rho)$.
\end{proof}

\begin{prob}{(S20.7)}{S20.7}
    A topological space is called \emph{totally disconnected} if every pair of points is contained in a pair of disjoint open sets whose union is the whole space. Prove that every countable metric space is totally disconnected.
\end{prob}

\begin{proof}
    Let $X$ be a countable metric space. Choose any distinct points $x,y \in X$ and let $\delta = d(x,y)$. Since $x \neq y$, $\delta > 0$. Because the interval $(0,\delta)$ is uncountable and $X$ is countable, there exists $r \in (0,\delta)$ such that no point in $X$ is distance $r$ from $x$. That is, for each $x_k \in X$, $d(x,x_k) \neq r$. Let $U = B_r(x)$ be the open ball of radius $r$ centered at $x$. Since there are no points exactly distance $r$ from $x$, $\ol{B_r(x)} = B_r(x)$. Therefore, $V = X \sm \ol{B_r(x)}$ is an open set containing $y$. Since $U$ and $V$ are disjoint and open such that $X = U \cup V$, it follows that $X$ is totally disconnected.

\end{proof}

\begin{prob}{(S12.7),(S02.4), (S00.6)}{S12.7}
    Define a metric $d$ on $N = \N \cup \{0\}$ by
        \[d(x,y) = 0\]
    whenever $x = y$ and otherwise
        \[d(x,y) = 5^{-k}\]
    where $5^k$ is the largest power of 5 that divides $|x-y|$.

    \begin{enumerate}[(a)]
        \item Verify that $d$ is a metric.
        \item Give an example of a sequence that converges to 0.
        \item Prove or disprove: The space $(N, d)$ is compact.
        \item Prove or disprove: The set of prime numbers greater than 103 is open in $(N,d)$.
    \end{enumerate}
\end{prob}

\begin{proof}
    For any $k \in \N$, $5^{-k} > 0$ and therefore $d(x,y) \geq 0$ for any $x, y \in N$. Clearly $d(x,y) = 0$ if and only if $x = y$, by definition of $d$. Since addition is commutative in $N$, $d(x,y) = d(y,x)$ for any $x,y \in N$. Now let $x,y,z \in N$ be arbitrary. If $x = z$, then $d(x,z) = 0 \leq d(x,y) + d(y,z)$. Otherwise $d(x,z) = 5^{-k}$ where $5^k$ is the largest power of 5 that divides $|x-z|$. Notice that $|x-z| \leq |x-y| + |y - z|$. Therefore, no higher power of 5 can divide either of $|x-y|$ or $|y-z|$. From here it follows that $d(x,z) \leq d(x,y) + d(y,z)$, as desired.
\end{proof}

\begin{solution}
    A trivial example is the constant sequence of all zeroes. Alternatively, consider the sequence $\{5^n\}$. Since $\frac{1}{5^n} \to 0$, $\{5^n\} \to 0$.
\end{solution}

\begin{solution}
    This space is not compact. Consider the sequence $\{1, 1 + 5, 1 + 5 + 5^2, \ldots\}$.  Since $d(5,0) = \frac{1}{5}$, the geometric series $\sum_{k=0}^\infty 5^k$ converges to $\frac{1}{1 -5} \not\in N$. In a metric space, any compact space must be sequentially compact. Therefore $N$ is not compact.
\end{solution}

\begin{solution}
    Let $A$ denote the set of all prime numbers greater than 103. Let $p \in A$ be arbitrary. For any $n \in \N$, $p + 5^n$ is the sum of two odd numbers and is therefore even. In particular, each $p + 5^n \not\in A$. For any $\epsilon > 0$, there exists sufficiently large $N$ such that $d(p + 5^N, p) = 5^{-N} < \epsilon$. But, $p + 5^N \not\in A$ and therefore no $\epsilon$ ball about $p$ is contained in $A$. That is, $A$ is not open.
\end{solution}

\begin{prob}{(S19.2)}{S19.2}
    Let $X$ be a compact metric space with metric $d$. Prove that for every $\epsilon > 0$ there exists a finite number of points $p_1,\ldots, p_n$ such that

    \begin{itemize}
        \item $X = \bigcup_{k=1}^n B_\epsilon(p_k)$
        \item $B_{\epsilon/4}(p_i) \cap B_{\epsilon/4}(p_j) = \varnothing$ whenever $i \neq j$.
    \end{itemize}
\end{prob}


\begin{proof}
    The proof here is nearly identical to that in (\ref{prob:S20.1}), replacing "3" with $\epsilon$ and "1" with $\epsilon/4$.
\end{proof}

\begin{prob}{(S20.1)}{S20.1}
    Let $X$ be a compact metric space. Prove that there exists a finite set of points $x_1, \ldots, x_n$ such that every point in $X$ is distance less than 3 from some $x_i$ and $d(x_i,x_j) \geq 1$ for any $i \neq j$.
\end{prob}

\begin{proof}
    Assume that $X$ is compact. Because $X$ is a metric space, this implies that $X$ is sequentially compact (\ref{thm:seqCompactMS}).\\

    Let $x_1 \in X$ be arbitrary and let $B_1 = B_3(x_1)$ be the open ball of radius 3 centered at $x_1$. If $X \sm B_1$ is empty, then the result holds. Otherwise, choose $x_2 \in X\sm B_1$ and define $B_2 = B_3(x_2)$. If $X \sm (B_1 \cup B_2)$ is empty, the result holds. Otherwise, choose $x_3 \in X \sm (B_1 \cup B_2)$. For each $n \in \N$, choose $x_{n+1} \in X \sm \bigcup_{k=1}^n B_n$ and define $B_{n+1} = B_3(x_{n+1})$. If at any step, no such point exists, the result holds.\\

    Assume now that this process can be repeated infinitely many times and consider the constructed sequence $\{x_n\}$. By construction, $d(x_i,x_j) \geq 1$ whenever $i \neq j$ meaning that it is impossible for any subsequence to converge. This contradicts $X$ being sequentially compact and therefore after $N$ steps the process must terminate. By construction, this implies the desired result.
\end{proof}

\begin{prob}{(S20.2)}{S20.2}
    Suppose that $X$ is a metric space such that every sequence in $X$ has a Cauchy subsequence. Prove that $X$ can be covered by finitely many balls of radius 1.
\end{prob}

\begin{proof}
    Suppose not. That is, assume that no finite collection of balls of radius 1 can cover $X$. Then we may construct a sequence of points in $X$ as follows: let $r = 1$ and let $x_1$ be any point in $X$. Choose $x_2 \in X \sm B_r(x_1)$. Such an $x_2$ must exist or else $X$ would be covered by one ball of radius 1. For each $n \in \N$, choose $x_n \in X \sm \bigcup_{k=1}^{n-1}B_r(x_k)$. If this process were to terminate after $n$ steps, then a finite number of balls of radius 1 would cover $X$.\\

    By assumption, the sequence $\{x_n\}$ must have some Cauchy subsequence. However, this is impossible since each of the $x_k$ are at least distance 1 apart.
\end{proof}

\begin{prob}{(F18.2)}{F18.2}
    Let $d: \Z^+ \times \Z^+ \to \R$ be the function
        \[d(x,y) = \begin{cases} 0 & x = y\\
            \frac{1}{x} + \frac{1}{y} & x \neq y
        \end{cases}.\]
    Prove that $\Z^+$ is a metric space with respect to $d$, but is not complete.
\end{prob}

\begin{proof}
    By definition, $d(x,x) = 0$. If $d(x,y) = 0$, then either $x = y$ or $\frac{1}{x} + \frac{1}{y} = -0$. But the second is impossible since both terms in the sum are positive. Therefore $d(x,y) = 0$ implies that $x = y$. Since addition is commutative, it's clear that $d(x,y) = d(y,x)$.\\

    Now let $x,y,z \in \Z^+$. If $x = y = z$, then it's clear that $d(x,z) = 0 \leq 0 + 0 = d(x,y) + d(y,z)$. Now suppose that $x = y$, but $y \neq z$. Then,
        \[d(x,z) \leq d(x,z) + 0 = d(x,y) + d(y,z).\]
    Finally assume that $x,y,z$ are all distinct. Then,
        \[d(x,z) = \frac{1}{x} + \frac{1}{z} \leq \frac{1}{x} + \frac{1}{y} + \frac{1}{y} + \frac{1}{z} = d(x,y) + d(y,z).\]
    The above properties demonstrate that $d$ is indeed a metric on $\Z^+$.\\

    To see that $(\Z^+, d)$ is not complete, consider the sequence $\{1,2,3,\ldots \}$. This sequence is Cauchy since as $m,n \to \infty$, $d(m,n) = \frac{1}{m} + \frac{1}{n} \to 0$. However, every subsequence  is unbounded and therefore cannot converge.
\end{proof}

\begin{prob}{(F19.5), (S02.2)}{F19.5}
    Define a $K$-contraction mapping of a metric space. Show that if $K < 1$, then a $K$ contraction of a complete metric space has a unique fixed point. Must this be true when $K = 1$?\\
    
    Let $f: X \to X$ and suppose there exists an $n \in \N$ and $K < 1$ where $f^{(n)}(x)$ is a $K$-contraction. Prove that $f$ has a unique fixed point.
\end{prob}

\begin{proof}
    To see that any $K$-contraction with $K < 1$ has a unique fixed point, see \ref{thm:contraction}.\\

    If $K = 1$, a $K$-contraction mapping need not have a unique fixed point. Let $f: \R \to \R$ be the identity map. Clearly $f$ is a contraction and $\R$ is complete. However, every point is fixed by $f$, violating the uniqueness.\\

    Suppose now that $f^{(n)}(x)$ is a $K$-contraction. Then there exists a unique point $x \in X$ such that $f^{(n)}(x) = x$. But,
        \[f(x) = f\left(f^{(n)}(x)\right) = f^{(n)}\big(f(x)\big)\]
    meaning that $f(x) = x$, by the uniqueness of the fixed point for $f^{(n)}$.
\end{proof}

\begin{prob}{(F16.3), (F06.4), (F18.4)}{F16.3}
    Prove that a metric space is compact if and only if it is sequentially compact.
\end{prob}

\begin{proof}
    See \ref{thm:seqCompactMS}.
\end{proof}

\begin{prob}{(F17.1), (F05.4)}{F05.4}
    Define what it means for a function $f: X \to Y$ to be continuous. Give the $\epsilon-\delta$ definition of continuity for metric spaces. Prove that these definitions are equivalent in a metric space.
\end{prob}

\begin{enumerate}[(1)]
    \item A function $f: X \to Y$ is continuous if for each open set $U \sq Y$, the set $f\inv(U)$ is open in $X$. 
    \item In a metric space, $f$ is continuous at $x \in X$ if for every $\epsilon > 0$ there exists $\delta > 0$ such that $d_X(x,y) < \delta$ implies that $d_Y(f(x),f(y)) < \epsilon$. The function $f$ is continuous if $f$ is continuous at each $x \in X$.
\end{enumerate}

\begin{proof}
    Let $(X,d)$ and $(Y,\rho)$ be metric spaces. Assume that (1) holds. Let $\epsilon > 0$ and $x \in X$. Then $B_\epsilon(f(x))$ is an open set in $Y$ and therefore $f\inv\left(B_\epsilon(f(x))\right)$ must be an open set in $X$. Since $(X,d)$ has a basis consisting of open balls and $x \in f\inv\left(B_\epsilon(f(x))\right)$, there exists some open ball $B_\delta(x) \sq X$ such that $f\left(B_\delta(x)\right) \sq B_\epsilon(f(x))$.\\

    Assume now that (2) holds and let $U$ be an open set in $Y$. Since the collection of open balls in $Y$ forms a basis for the topology, it suffices to show that the preimage of any open ball in $Y$ is open in $X$. Therefore without loss of generality, assume that $B_\epsilon(y)$ is an open ball in $Y$. Let $x \in f\inv\left(B_\epsilon(y)\right)$. Then $f(x) \in B_\epsilon(y)$. Since $B_\epsilon(y)$ is open, there exists $\epsilon'$ such that $B_{\epsilon'}(f(x)) \sq B_\epsilon(y)$. Choose $\delta > 0$ such that $f\left(B_\delta(x)\right) \sq B_{\epsilon'}(f(x))$. Then $B_\delta(x)$ is an open set in $X$ containing $x$ such that $B_\delta(x) \sq f\inv\left(B_\epsilon(y)\right)$. Therefore, $f\inv\left(B_\epsilon(y)\right)$ is open, as desired.
\end{proof}

\begin{prob}{F13.5}{F13.5}
    Let $X$ be a complete metric space and $\{C_n\}_{n \in \N}$ a collection of non-empty closed sets such that $C_1 \supseteq C_2 \supseteq \cdots$. Assume that the sequence of diameters of the $C_n$ goes to zero. Prove that the intersection $\cap C_n$ of this collection is nonempty.
\end{prob}

\begin{proof}
    Construct a sequence $\{x_n\}$ by choosing any $x_i \in C_i$ for each $i = 1,2,\ldots$. Because the sets are nested, $x_n \in C_k$ whenever $k \leq n$ for each $n \in \N$.\\

    Let $\{r_n\}$ be the sequence of diameters of the $C_n$. By assumption, $r_n \to 0$. Let $\epsilon > 0$ be arbitrary and choose $N \in \N$ where $n \geq N$ implies that $|r_n| < \epsilon$. Assume that $m,n \geq N$ and that $m \geq n$. Then,
        \[\norm[x_n - x_m] \leq r_n < \epsilon \]
    since $x_n,x_m \in C_n$. This means that $\{x_n\}$ is a Cauchy sequence in a complete space -- let $x \in X$ be the limit of $\{x_n\}$.\\

    To see that $x \in C_N$ for each $N$, notice that $\{x_n\}_{n \geq N}$ is a subseqeunce of $\{x_n\}$ that is contained in $C_N$. Since $x_n \to x$, this subseqeunce also converges to $x$ meaning that $x$ is a limit point of $C_N$. But, $C_N$ is closed and therefore contains all its limit points. Since $x \in \bigcap_{n=1}^\infty C_n$, the intersection is nonempty.
\end{proof}

\begin{prob}{S12.4}{S12.4}
    Suppose that $(X,d)$ is a metric space and $A \sq X$.
    \begin{enumerate}[(a)]
        \item For a fixed $x \in X$, define what is meant by $d(x,A)$.
        \item Show that for all $x,y \in X$, $d(x,A) \leq d(x,y) + d(y,A)$.
        \item Show that the function $f: X \to \R$ given by $f(x) = d(x,A)$ is a continuous function.
    \end{enumerate}
\end{prob}

Fix $x \in X$. Then $d(x,A) = \inf_{a \in A} d(x,a)$ describes the distance from $x$ to the set $A$.

\begin{proof}
    Let $x,y \in X$ be arbitrary. Because $d$ is a metric, for each $a \in A$, $d(x,a) \leq d(x,y) + d(y,a)$. Therefore,
        \[d(x,A) \leq d(x,a) \leq d(x,y) + d(y,a). \]
    This means that for each $a \in A$, $d(x,A) - d(x,y) \leq d(y,a)$. Because $d(y,A)$ is the infimum over all $d(y,a)$ with $a \in A$, it is the greatest lower bound. It then follows that $d(x,A) - d(x,y) \leq d(y,A)$, as desired.
\end{proof}

\begin{prob}{S20}{}
Let $(X,d)$ be a metric space and fix a point $x_0 \in X$. Let $\rho$ be a new metric given by $\rho(x,y) = d(x,x_0) + d(y,x_0)$ whenever $x \neq y$ and $\rho(x,y) = 0$ if $x = y$. Verify that $\rho$ is a metric and $(X,\rho)$ is complete.
\end{prob}

\begin{proof}
By construction, $\rho(x,y) \geq 0$ for each $x,y \in X$. Suppose $\rho(x,y) = 0$ but $x \neq y$. Then, $0 = \rho(x,y) = d(x,x_0) + d(y,y_0)$. Since at most one of $x$ and $y$ can be $x_0$, $d(x,x_0) + d(y,y_0) > 0$. Therefore $\rho(x,y) = 0$ if and only if $x = y$. Suppose now that $x,y, z \in X$. Then,
    \[\rho(x,y) + \rho(y,z) = d(x,x_0) + d(y,x_0) + d(y,x_0) + d(z,x_0) = \rho(x,z) + 2 d(y,x_0) \geq \rho(x,z) \]
proving that $\rho$ is a metric.\\

To see that $(X, \rho)$ is a complete metric space, let $(x_n)$ be a Cauchy sequence in $(X, \rho)$. Let $\epsilon > 0$ and choose $N \in \N$ sufficiently large such that $\rho(x_N, x_n) < \epsilon$ whenever $n \geq N$. This means that whenever $n \geq N$,
    \[ d(x_n,x_0) \leq d(x_N, x_0) + d(x_n, x_0) = \rho(x_N, x_m) < \epsilon. \]
Therefore, $x_n \to x_0$ in $(X,d)$. Equivalently, as $n \to \infty$, $d(x_n,x_0) \to 0$. Then,
    \[\rho(x_n, x_0) = d(x_n, x_0) + d(x_0, x_0) = d(x_n, x_0) \]
meaning that as $n \to \infty$, $\rho(x_n,x_0) \to 0$. That is, $x_n \to x_0$ in $(X,\rho)$.
\end{proof}



\end{document}