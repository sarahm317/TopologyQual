\documentclass[../TopologyQualSolutions.tex]{subfiles}

\begin{document}

\section{Metric Spaces}

\begin{prob}{(S20.2)}{S20.2}
    Suppose that $X$ is a metric space such that every sequence in $X$ has a Cauchy subsequence. Prove that $X$ can be covered by finitely many balls of radius 1.
\end{prob}

\begin{proof}
    Suppose not. That is, assume that no finite collection of balls of radius 1 can cover $X$. Then we may construct a sequence of points in $X$ as follows: let $r = 1$ and let $x_1$ be any point in $X$. Choose $x_2 \in X \sm B_r(x_1)$. Such an $x_2$ must exist or else $X$ would be covered by one ball of radius 1. For each $n \in \N$, choose $x_n \in X \sm \bigcup_{k=1}^{n-1}B_r(x_k)$. If this process were to terminate after $n$ steps, then a finite number of balls of radius 1 would cover $X$.\\

    By assumption, the sequence $\{x_n\}$ must have some Cauchy subsequence. However, this is impossible since each of the $x_k$ are at least distance 1 apart.
\end{proof}

\begin{prob}{(F18.2)}{F18.2}
    Let $d: \Z^+ \times \Z^+ \to \R$ be the function
        \[d(x,y) = \begin{cases} 0 & x = y\\
            \frac{1}{x} + \frac{1}{y} & x \neq y
        \end{cases}.\]
    Prove that $\Z^+$ is a metric space with respect to $d$, but is not complete.
\end{prob}

\begin{proof}
    By definition, $d(x,x) = 0$. If $d(x,y) = 0$, then either $x = y$ or $\frac{1}{x} + \frac{1}{y} = -0$. But the second is impossible since both terms in the sum are positive. Therefore $d(x,y) = 0$ implies that $x = y$. Since addition is commutative, it's clear that $d(x,y) = d(y,x)$.\\

    Now let $x,y,z \in \Z^+$. If $x = y = z$, then it's clear that $d(x,z) = 0 \leq 0 + 0 = d(x,y) + d(y,z)$. Now suppose that $x = y$, but $y \neq z$. Then,
        \[d(x,z) \leq d(x,z) + 0 = d(x,y) + d(y,z).\]
    Finally assume that $x,y,z$ are all distinct. Then,
        \[d(x,z) = \frac{1}{x} + \frac{1}{z} \leq \frac{1}{x} + \frac{1}{y} + \frac{1}{y} + \frac{1}{z} = d(x,y) + d(y,z).\]
    The above properties demonstrate that $d$ is indeed a metric on $\Z^+$.\\

    To see that $(\Z^+, d)$ is not complete, consider the sequence $\{1,2,3,\ldots \}$. This sequence is Cauchy since as $m,n \to \infty$, $d(m,n) = \frac{1}{m} + \frac{1}{n} \to 0$. However, every subsequence  is unbounded and therefore cannot converge.
\end{proof}

\begin{prob}{(F19.5)}{F19.5}
    Define a $K$-contraction mapping of a metric space. Show that if $K < 1$, then a $K$ contraction of a complete metric space has a unique fixed point. Must this be true when $K = 1$?\\
    
    Let $f: X \to X$ and suppose there exists an $n \in \N$ and $K < 1$ where $f^{(n)}(x)$ is a $K$-contraction. Prove that $f$ has a unique fixed point.
\end{prob}

\begin{proof}
    To see that any $K$-contraction with $K < 1$ has a unique fixed point, see \ref{thm:contraction}.\\

    If $K = 1$, a $K$-contraction mapping need not have a unique fixed point. Let $f: \R \to \R$ be the identity map. Clearly $f$ is a contraction and $\R$ is complete. However, every point is fixed by $f$, violating the uniqueness.\\

    Suppose now that $f^{(n)}(x)$ is a $K$-contraction. Then there exists a unique point $x \in X$ such that $f^{(n)}(x) = x$. But,
        \[f(x) = f\left(f^{(n)}(x)\right) = f^{(n)}\big(f(x)\big)\]
    meaning that $f(x) = x$, by the uniqueness of the fixed point for $f^{(n)}$.
\end{proof}

\begin{prob}{(F16.3), (F06.4), (F18.4)}{F16.3}
    Prove that a metric space is compact if and only if it is sequentially compact.
\end{prob}

\begin{proof}
    See \ref{thm:seqCompactMS}.
\end{proof}

\begin{prob}{F05.4}{F05.4}
    Define what it means for a function $f: X \to Y$ to be continuous. Give the $\epsilon-\delta$ definition of continuity for metric spaces. Prove that these definitions are equivalent in a metric space.
\end{prob}

\begin{enumerate}[(1)]
    \item A function $f: X \to Y$ is continuous if for each open set $U \sq Y$, the set $f\inv(U)$ is open in $X$. 
    \item In a metric space, $f$ is continuous at $x \in X$ if for every $\epsilon > 0$ there exists $\delta > 0$ such that $d_X(x,y) < \delta$ implies that $d_Y(f(x),f(y)) < \epsilon$. The function $f$ is continuous if $f$ is continuous at each $x \in X$.
\end{enumerate}

\begin{proof}
    Let $(X,d)$ and $(Y,\rho)$ be metric spaces. Assume that (1) holds. Let $\epsilon > 0$ and $x \in X$. Then $B_\epsilon(f(x))$ is an open set in $Y$ and therefore $f\inv\left(B_\epsilon(f(x))\right)$ must be an open set in $X$. Since $(X,d)$ has a basis consisting of open balls and $x \in f\inv\left(B_\epsilon(f(x))\right)$, there exists some open ball $B_\delta(x) \sq X$ such that $f\left(B_\delta(x)\right) \sq B_\epsilon(f(x))$.\\

    Assume now that (2) holds and let $U$ be an open set in $Y$. Since the collection of open balls in $Y$ forms a basis for the topology, it suffices to show that the preimage of any open ball in $Y$ is open in $X$. Therefore without loss of generality, assume that $B_\epsilon(y)$ is an open ball in $Y$. Let $x \in f\inv\left(B_\epsilon(y)\right)$. Then $f(x) \in B_\epsilon(y)$. Since $B_\epsilon(y)$ is open, there exists $\epsilon'$ such that $B_{\epsilon'}(f(x)) \sq B_\epsilon(y)$. Choose $\delta > 0$ such that $f\left(B_\delta(x)\right) \sq B_{\epsilon'}(f(x))$. Then $B_\delta(x)$ is an open set in $X$ containing $x$ such that $B_\delta(x) \sq f\inv\left(B_\epsilon(y)\right)$. Therefore, $f\inv\left(B_\epsilon(y)\right)$ is open, as desired.
\end{proof}

\begin{prob}{F13.5}{F13.5}
    Let $X$ be a complete metric space and $\{C_n\}_{n \in \N}$ a collection of non-empty closed sets such that $C_1 \supseteq C_2 \supseteq \cdots$. Assume that the sequence of diameters of the $C_n$ goes to zero. Prove that the intersection $\cap C_n$ of this collection is nonempty.
\end{prob}

\begin{proof}
    Construct a sequence $\{x_n\}$ by choosing any $x_i \in C_i$ for each $i = 1,2,\ldots$. Because the sets are nested, $x_n \in C_k$ whenever $k \leq n$ for each $n \in \N$.\\

    Let $\{r_n\}$ be the sequence of diameters of the $C_n$. By assumption, $r_n \to 0$. Let $\epsilon > 0$ be arbitrary and choose $N \in \N$ where $n \geq N$ implies that $|r_n| < \epsilon$. Assume that $m,n \geq N$ and that $m \geq n$. Then,
        \[\norm[x_n - x_m] \leq r_n < \epsilon \]
    since $x_n,x_m \in C_n$. This means that $\{x_n\}$ is a Cauchy sequence in a complete space -- let $x \in X$ be the limit of $\{x_n\}$.\\

    To see that $x \in C_N$ for each $N$, notice that $\{x_n\}_{n \geq N}$ is a subseqeunce of $\{x_n\}$ that is contained in $C_N$. Since $x_n \to x$, this subseqeunce also converges to $x$ meaning that $x$ is a limit point of $C_N$. But, $C_N$ is closed and therefore contains all its limit points. Since $x \in \bigcap_{n=1}^\infty C_n$, the intersection is nonempty.
\end{proof}

\begin{prob}{S12.4}{S12.4}
    Suppose that $(X,d)$ is a metric space and $A \sq X$.
    \begin{enumerate}[(a)]
        \item For a fixed $x \in X$, define what is meant by $d(x,A)$.
        \item Show that for all $x,y \in X$, $d(x,A) \leq d(x,y) + d(y,A)$.
        \item Show that the function $f: X \to \R$ given by $f(x) = d(x,A)$ is a continuous function.
    \end{enumerate}
\end{prob}

Fix $x \in X$. Then $d(x,A) = \inf_{a \in A} d(x,a)$ describes the distance from $x$ to the set $A$.

\begin{proof}
    Let $x,y \in X$ be arbitrary. Because $d$ is a metric, for each $a \in A$, $d(x,a) \leq d(x,y) + d(y,a)$. Therefore,
        \[d(x,A) \leq d(x,a) \leq d(x,y) + d(y,a). \]
    This means that for each $a \in A$, $d(x,A) - d(x,y) \leq d(y,a)$. Because $d(y,A)$ is the infimum over all $d(y,a)$ with $a \in A$, it is the greatest lower bound. It then follows that $d(x,A) - d(x,y) \leq d(y,A)$, as desired.
\end{proof}

\begin{prob}{S20}{}
Let $(X,d)$ be a metric space and fix a point $x_0 \in X$. Let $\rho$ be a new metric given by $\rho(x,y) = d(x,x_0) + d(y,x_0)$ whenever $x \neq y$ and $\rho(x,y) = 0$ if $x = y$. Verify that $\rho$ is a metric and $(X,\rho)$ is complete.
\end{prob}

\begin{proof}
By construction, $\rho(x,y) \geq 0$ for each $x,y \in X$. Suppose $\rho(x,y) = 0$ but $x \neq y$. Then, $0 = \rho(x,y) = d(x,x_0) + d(y,y_0)$. Since at most one of $x$ and $y$ can be $x_0$, $d(x,x_0) + d(y,y_0) > 0$. Therefore $\rho(x,y) = 0$ if and only if $x = y$. Suppose now that $x,y, z \in X$. Then,
    \[\rho(x,y) + \rho(y,z) = d(x,x_0) + d(y,x_0) + d(y,x_0) + d(z,x_0) = \rho(x,z) + 2 d(y,x_0) \geq \rho(x,z) \]
proving that $\rho$ is a metric.\\

To see that $(X, \rho)$ is a complete metric space, let $(x_n)$ be a Cauchy sequence in $(X, \rho)$. Let $\epsilon > 0$ and choose $N \in \N$ sufficiently large such that $\rho(x_N, x_n) < \epsilon$ whenever $n \geq N$. This means that whenever $n \geq N$,
    \[ d(x_n,x_0) \leq d(x_N, x_0) + d(x_n, x_0) = \rho(x_N, x_m) < \epsilon. \]
Therefore, $x_n \to x_0$ in $(X,d)$. Equivalently, as $n \to \infty$, $d(x_n,x_0) \to 0$. Then,
    \[\rho(x_n, x_0) = d(x_n, x_0) + d(x_0, x_0) = d(x_n, x_0) \]
meaning that as $n \to \infty$, $\rho(x_n,x_0) \to 0$. That is, $x_n \to x_0$ in $(X,\rho)$.
\end{proof}



\end{document}