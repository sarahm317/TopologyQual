\documentclass[../TopologyQualSolutions.tex]{subfiles}

\begin{document}

\section{Results to Memorize}

\begin{prop}{}{ctsImage}
    \begin{enumerate}[(a)]
    \item The continuous image of a compact space is compact.
    \item The continuous image of a connected space is connected.
    \item   The continuous image of a path-connected space is path-connected.
    \end{enumerate}
\end{prop}

\begin{proof}
    Let $f: X \to Y$ be continuous and suppose that $X$ is compact. Suppose that $\{U_\alpha\}$ is an open cover for $f(X)$. Since $f$ is continuous, each $f\inv(U_\alpha)$ is open in $X$. For any $x \in X$, $f(x) \in U_\alpha$ for some $U_\alpha$. Therefore, $x \in f\inv(U_\alpha)$ implying that $\{f\inv(U_alpha)\}$ is an open cover for $X$. Since $X$ is compact, extract a finite subcover, say $\{f\inv(U_1),\ldots, f\inv(U_n)\}$. Consider the corresponding collection $\{U_1,\ldots, U_n\}$ from the original cover. For each $k$, $f\left(f\inv(U_k)\right) \sq U_k$. Since $\{f\inv(U_1),\ldots, f\inv(U_n)\}$ covers $X$, $\{U_1,\ldots, U_n\}$ covers $f(X)$, as desired.
\end{proof}

\begin{proof}
    Suppose that $f: X \to Y$ is continuous and $X$ is connected. Seeking a contradiction, let $U \cup V = f(X)$ be a separation for the image of $X$. Since $f$ is continuous, both $f\inv(U)$ and $f\inv(V)$ are open in $X$. Since $U \cup V = f(X)$, each $x \in X$ is contained in either $f\inv(U)$ or $f\inv(V)$. Therefore $X \sq f\inv(U) \cup f\inv(V)$. Trivially, $f\inv(U) \cup f\inv(V) \sq X$ and so $X = f\inv(U) \cup f\inv(V)$. Since both $f\inv(U)$ and $f\inv(V)$ are nonempty and $X$ is connected, $f\inv(U)\cap f\inv(V) \neq \varnothing$. This implies $U \cap V \neq \varnothing$.
\end{proof}

\begin{proof}
    Suppose that $f: X \to Y$ is continuous and $X$ is path-connected. Let $f(x),f(y)$ be in the image of $f$. Since $X$ is path-connected, there exists a path $\gamma: [0,1] \to X$ from $x$ to $y$ where $\gamma(0) = x$ and $\gamma(y) = 1$. Let $\alpha = f\circ \gamma: [0,1] \to f(X)$. Then $\alpha$ is the composition of continuous functions and so also must be continuous. Also, $\alpha(0) = f(x)$ and $\alpha(1) = f(y)$ meaning that $\alpha$ is a path from $f(x)$ to $f(y)$.
\end{proof}

\begin{prop}{}{productResults}
    \begin{enumerate}[(a)]
    \item The product space $X \times Y$ is compact if and only if both $X$ and $Y$ are compact.
    \item The product space $X \times Y$ is connected if and only if both $X$ and $Y$ are connected.
    \item The product space $X \times Y$ is path-connected if and only if both $X$ and $Y$ are path-connected.
    \item The product space $X \times Y$ is Hausdorff if and only if both $X$ and $Y$ are Hausdorff.
    \end{enumerate}
\end{prop}

\begin{proof}
    Assume first that $X \times Y$ is compact. Since $p_X$ and $p_Y$ are both continuous maps from $X \times Y$ onto $X$ and $Y$ respectively, it follows that $X$ and $Y$ are both compact.\\

    Now assume that $X$ and $Y$ are both compact. Fix $x_0 \in X$ and let $N \sq X \times Y$ be any open subset such that $\{x_0\} \times Y \sq N$.

    \begin{claim}

        Let $N$ be any open set containing $\{x_0\} \times Y$. Then there exists an open set $W \sq X$ containing $x_0$ such that $W \times Y \sq N$.

        \begin{proof}
            Let $\{U_\alpha \times V_\alpha\}$ be an open cover for $\{x_0\} \times Y$ consisting of basis elements for the product topology, each $U_\alpha \times V_\alpha \sq N$. Because $\{x_0\} \times Y \cong Y$ and $Y$ is compact, $\{x_0\} \times Y$ is also compact. Therefore, there exists a finite subcover, say $U_1 \times V_1, \ldots, U_n \times V_n$. Define $W = \bigcap_{k=1}^n U_k$, removing if necessary any $U_j$ that does not contain $x_0$. Then $W$ is open and contains $x_0$. Furthermore, $W \times Y \sq \bigcup_{k=1}^n U_k \times V_k$: if $(x,y) \in W \times Y$, then $x \in U_j$ for each $j$ and $y \in Y$ and thus is in some $V_i$. Since each $U_k\times V_k \sq N$, it follows that $W \times Y \sq \bigcup_{k=1}^n U_k \times V_k \sq N$, as desired.
        \end{proof}
    \end{claim}

    Let $\scr{W} = \{W_\alpha\}$ be an arbitrary open cover of $X \times Y$. For a fixed $x_0$, the subspace $\{x_0\} \times Y$ is compact and thus may be covered by finitely many $W_\alpha$, say $W_1,\ldots, W_n$. Define $N = W_1 \cup \cdots \cup W_n$. Then $N$ is an open neighborhood about $\{x_0\} \times Y$ and by the claim there exists an open set $U \sq X$ such that
        \[\{x_0\} \times Y \sq U \times Y \sq N.\]
    Notice that this implies $U \times Y$ can be covered by finitely many $W_\alpha$. Repeat this process for each $x \in X$ to obtain an open set $U_x$ containing $x$ such that $U_x \times Y$ can be covered by finitely many $W_\alpha$.\\

    The collection $\{U_x\}_{x \in X}$ then forms an open cover of $X$. Since $X$ is compact, extract a finite subcover, say $U_{x_1}, \ldots, U_{x_m}$. Each $U_{x_j} \times Y$ can be covered by finitely many $W_\alpha$, by construction. By concatenating the finitely many $W_\alpha$ needed to cover each of the finitely many $U_{x_j} \times Y$, we obtain a finite subcover of $\scr{W}$ for $X \times Y$.
\end{proof}

\begin{proof}
    Suppose first that $X \times Y$ is connected. Since the projection map $p_X: X\times Y \to X$ is both surjective and continuous, and the continuous image of a connected set is connected, $X$ is connected. Likewise, $Y$ is connected.\\

    Now assume that $X$ and $Y$ are both connected sets. Suppose that $A$ and $B$ are nonempty, disjoint, open subsets of $X \times Y$ such that $X \times Y = A \cup B$. Fix $y \in Y$ and notice that $X \cong X \times \{y\}$. Since $X$ is connected and homeomorphisms preserve connectedness, $X \times \{y\}$ must also be connected. Therefore, without loss of generality, $X \times \{y\} \sq A$. If this were not the case, by writing $A$ and $B$ as unions of basic open sets we would obtain a separation for $X$. Similarly, for a fixed $x \in X$, $Y \cong \{x\} \times Y$. Since $Y$ is connected and $(x,y) \in U$, it must be the case that $\{x\} \times Y \sq A$. But this would imply that $X \times Y \sq A$, contradicting the choice of $A$ and $B$.
\end{proof}

\begin{proof}
    Assume that $X \times Y$ is path-connected. Since the projection maps $p_X$ and $p_Y$ are continuous surjections onto $X$ and $Y$ respectively, both $X$ and $Y$ are the continuous images of path-connected sets and are therefore path-connected.\\

    Now assume that both $X$ and $Y$ are path-connected. Let $(a,b)$ and $(x,y)$ be arbitrary points in $X \times Y$. Because $X$ is path-connected, there exists a path $\gamma: [0,1] \to X$ such that $\gamma(0) = a$ and $\gamma(1) = x$. Likewise, since $Y$ is path-connected, there exists a path $\alpha: [0,1] \to Y$ such that $\alpha(0) = b$ and $\alpha(1) = y$. Define $f: [0,1] \to X \times Y$ by 
        \[f(t) = (\gamma(t), \alpha(t)).\]
    Clearly $f(0) = (a,b)$ and $f(1) = (x,y)$. Since both the component functions of $f$ are continuous, $f$ is continuous and is thus a path between $(a,b)$ and $(x,y)$.
\end{proof}

\begin{proof}
    Assume first that $X \times Y$ is Hausdorff. Let $a,b \in X$ be distinct points. Fix $y \in Y$. Because $X \times Y$ is Hausdorff, there exist disjoint, open sets $W_1$ and $W_2$ such that $(a,y) \in W_1$ and $(b,y) \in W_2$. By definition of the product topology, there exists a basic open set $U_1 \times V_1 \sq W_1$ that contains $(a,y)$. Similarly there exists a basic open set $U_2 \times V_2 \sq W_2$ that contains $(b,y)$. Then $U_1$ and $U_2$ are disjoint open sets in $X$ that contain $a$ and $b$, respectively. Showing that $Y$ is Hausdorff is analogous, fixing an element in $X$ instead.\\

    Assume now that both $X$ and $Y$ are Hausdorff. Let $(a,b)$ and $(x,y)$ be distinct points in $X \times Y$. Since $X$ is Hausdorff, choose disjoint open sets $U_1$ and $U_2$ in $X$ that contain $a$ and $x$, respectively. Similarly, choose disjoint open sets $V_1$ and $V_2$ that contain $b$ and $y$, respectively. Then $U_1 \times V_1$ and $U_2 \times V_2$ are open sets in $X \times Y$ that are disjoint and separate $(a,b)$ and $(x,y)$.
\end{proof}

\begin{prop}{}{compactHaus}
    A compact set in a Hausdorff space is closed.
\end{prop}

\begin{proof}
    Let $A \sq X$ be a compact subspace of a Hausdorff space. If $X - A = \varnothing$, $A$ is trivially closed. Otherwise, let $x \in X - A$. For each $y \in A$, choose nonempty, disjoint, open sets $U_y$ and $V_y$ such that $x \in U_y$ and $y \in V_y$. Then the collection $\{V_y\}$ is an open cover for $A$. Since $A$ is compact there exists a finite subcover, say $\{V_1,\ldots, V_n\}$. Let $\{U_1,\ldots, U_n\}$ be the open sets that correspond to the chosen $V_k$. Let $U = \bigcap_{k=1}^n U_k$. Then $U$ is an open set containing $x$ that is  disjoint from each $V_y$. In particular, this means that $U \sq X - A$ and as $x \in X - A$ was arbitrary, it follows that $X - A$ is open. Whence $A$ is closed.
\end{proof}

\begin{prop}{}{compactClosed}
    A closed subspace of a compact set is compact.
\end{prop}

\begin{proof}
    Suppose that $A \sq X$ is a closed subspace of a compact set. Let $\{U_\alpha\}$ be an open cover of $A$. Since $A$ is closed, $X - A$ is open and therefore the collection $\{U_\alpha\} \cup \{X - A\}$ is an open cover for $X$. Because $X$ is compact, we may extract a finite subcover. If $X - A$ is in the finite subcover, removing it from the list yields a finite subcover for $A$, as desired.
\end{proof}

\begin{prop}{}{}
    A continuous bijection from a compact space to a Hausdorff space is a homeomorphism.
\end{prop}

\begin{proof}
    Suppose that $f: X \to Y$ is a continuous bijection from a compact space to a Hausdorff space. Let $g: Y \to X$ be the inverse of $f$. Let $A \sq X$ and notice that $g\inv(A) = f(A)$ since $f$ and $g$ are inverses. Therefore to show that $g$ is continuous, it suffices to show that $f(A)$ is closed for each closed subset $A$ of $X$.\\

    Let $A \sq X$ be closed. Then, $A$ is a closed subset of a compact set and therefore is compact (\ref{thm:compactClosed}). Since the continuous image of a compact set is compact (\ref{thm:ctsImage}), $f(A) \sq Y$ is compact. But, $Y$ is Hausdorff and since a compact set in a Hausdorff space is closed (\ref{thm:compactHaus}), $f(A)$ is closed.
\end{proof}

\begin{prop}{}{seqCompactMS}
    Suppose that $(X,d)$ is a metric space. Then $X$ is compact if and only if $X$ is sequentially compact.
\end{prop}

\begin{proof}

    Suppose first that $X$ is compact. Seeking a contraction, let $\{x_\}$ be a sequence in $X$ with no convergent subsequence. Then $\{x_n\}$ contains an infinite number of distinct points. Fix $x \in X$. Because no subsequence of $\{x_n\}$ converges, there exists $\epsilon_x > 0$ such that $B_{\epsilon_x}(x)$ contains finitely many terms of the sequence. Now consider the open cover of $X$ given by $\{B_{\epsilon_x}(x)\}_{x \in X}$. Since $X$ is compact, this cover must have a finite subcover. However, each element of the cover contains at most finitely many terms of the sequence and hence the finite subcover contains only finitely many terms of the sequence. But there are infinitely many distinct terms in $\{x_n\}$ meaning that this subcover does not cover $X$, a contradiction.\\

    Suppose now that $X$ is sequentially compact. Let $\scr{U}= \{U_\lambda\}_{\lambda \in \Lambda}$ be an arbitrary open cover of $X$.
    
    \begin{claim}
        There exists $\epsilon > 0$ such that for any $x \in X$, there exists $\lambda \in \Lambda$ where $B_\epsilon(x) \sq U_\lambda$.

        \begin{proof}
            Suppose otherwise. For each $n \in \N$, define $x_n \in X$ to be such that $B_{1/n}(x_n) \not\sq U_\lambda$ for any $\lambda \in \Lambda$. By assumption, $X$ is sequentially compact and therefore there exists some convergent subsequence, say $\{x_{n_k}\}$ of $\{x_n\}$. Assume that $x_{n_k} \to y \in X$. Since $\scr{U}$ is an open cover of $X$, there exists $\alpha \in \Lambda$ and $\epsilon' > 0$ such that $B_{\epsilon'}(y) \sq U_\alpha$. Since $x_{n_k} \to y$, there exists sufficiently large $K$ such that $x_{n_K} \in B_{\epsilon'}(y)$. Define
                \[r = \frac{\epsilon' - d(x_{n_K}, y)}{2}\]
            and observe that $B_r(x_{n_K}) \sq B_{\epsilon'}(y) \sq U_\alpha$. This contradicts the way in which the sequence $\{x_n\}$ was constructed.
        \end{proof}

    \end{claim}

    Let $\epsilon > 0$ be as guaranteed by the claim above. Let $\scr{B} = \{B_\epsilon(x)\}_{x \in X}$.

    \begin{claim}
        The open cover $\scr{B} = \{B_\epsilon(x)\}_{x \in X}$ has a finite subcover.

        \begin{proof}
            Suppose not. Let $z_1 \in X$ be arbitrary. Since $\scr{B}$ has no finite subcover, there exists $z_2 \in X \sm B_{\epsilon}(z_1)$. Recursively define a sequence $\{z_n\}$ by choosing $z_n \in X \sm \bigcup_{k=1}^{n-1}B_\epsilon(x_k)$. If this process were to terminate at any point, then $\scr{B}$ would have a finite subcover. By construction, the sequence $\{z_n\}$ has no convergent subsequence since for any $m,n \in \N$, $d(z_n,z_m) > \epsilon$. Since $X$ is sequentially compact, this is a contradiction.
        \end{proof}
    \end{claim}

    Choose a finite subcover, say $B_\epsilon(x_1),\ldots, B_\epsilon(x_N)$ of $\scr{B}$. Because each $B_\epsilon(x_j) \sq U_{\alpha_j}$ for some $\alpha_j \in \Lambda$, the list $U_{\alpha_1},\ldots, U_{\alpha_N}$ is a finite subcover of $\scr{U}$. Whence, $X$ is compact.  
\end{proof}

\begin{prop}{Contraction Mapping Theorem}{contraction}
    Let $X$ be a complete metric space and $f: X \to X$ a contraction map. Then $f$ has a unique fixed point.
\end{prop}

\begin{proof}
    Let $0 \leq \alpha < 1$ be such that
        \[d(f(x),f(y)) \leq \alpha d(x,y)\]
    for each $x,y \in X$. Fix $x \in X$ and define a sequence in $X$ by $x_n = f^{(n)}(x)$ where $f^{(n)}(x)$ denotes composition of $f$, $n$ times. Let $x_0 = x$. If $f(x) = x$, then $x$ is a fixed point of $f$. Suppose $f(x) \neq x$ so that $d(x,f(x)) > 0$.\\


    \begin{claim}
        $\{x_n\}$ is a Cauchy sequence in $X$.

        \begin{proof}
            Fix $\epsilon > 0$ and let $m, n \in \N$ where $m \geq n$. Let $k = m-n$. Observe:

            \begin{align*}
                d(x_n,x_m) &= d\left(f^{(n)}(x), f^{(m)}(x)\right)\\
                &\leq \alpha^{n}d\left(x, f^{(k)}(x)\right)
            \end{align*}

            by applying the contraction property $n$ times. Also notice that
            \begin{align*}
                d\left(x, f^{(k)}(x)\right) &\leq \sum_{j=0}^{k-1} d\left(f^{(j)}(x), f^{(j+1)}(x)\right)\\
                &\leq \sum_{j=0}^{k-1} \alpha^j d\left(x, f(x)\right)\\
                &= d\left(x, f(x)\right) \sum_{j=0}^{k-1} \alpha^j\\
            \end{align*}
            Therefore,
                   \[d(x_n,x_m) \leq \alpha^n d\left(x, f(x)\right) \sum_{j=0}^{k-1} \alpha^j = d\left(x, f(x)\right) \sum_{j=n}^{m-1} \alpha^j\]    
            Since $0 \leq \alpha < 1$, $\sum_{j=n}^{m-1} \alpha^j$ is the tail-end of a convergent geometric series. Therefore, by choosing sufficiently large $m,n$,
            \[d(x_n,x_m) \leq d\left(x, f(x)\right) \sum_{j=n}^{m-1} \alpha^j < \epsilon.\] 
        \end{proof}
    \end{claim}

    Since $\{x_n\}$ is a Cauchy sequence in a complete space, there exists a unique $y \in X$ such that $f^{(n)}(x) = x_n \to y$. Furthermore, any subsequence of $\{x_n\}$ also must converge to $y$. As $f$ is a contraction mapping, $f$ is also continuous and therefore,
    \[y = \lim_{n\to\infty} x_{n+1} = \lim_{n\to\infty} f(x_n) = f\left(\lim_{n\to\infty} x_n\right) = f(y).\]
That is, $y$ is a fixed point of $f$. Suppose now that $y' \in X$ is such that $f(y') = y'$. Then,
    \[d(y,y') \leq d(y,f(y)) + d(f(y),f(y')) + d(y',f(y')) = d(f(y), f(y')).\]
Since $f$ is a contraction mapping,
    \[d(f(y), f(y')) \leq \alpha d(y,y') < d(y,y')\]
which is a contradiction unless $d(y,y') = 0$. Therefore $y$ is the unique point in $X$ for which $f(y) = y$.
\end{proof}

\begin{prop}{}{}
    Let $C([0,1])$ be the collection of continuous functions from [0,1] to $\R$. Then $(C([0,1]), \norm_{\sup})$ is connected and complete.
\end{prop}

\begin{prop}{}{topSine}
    The  topologist's sine curve is connected but is not path-connected.
\end{prop}

\begin{proof}
    Define $S = \{(x,\sin(1/x): x> 0)\}$ and $Y = \{0\} \times [-1,1]$. Let $X = Y \cup S \sq \R^2$ be the topologist's sine curve.\\

    \begin{claim}
        The closure of $S$ in $X$ is $X$.
        \begin{proof}
            By definition of closure, $S \sq \ol{S}$. Suppose that $p = (0,y) \in Y$. We must show that $p$ is the limit of a sequence of points in $S$. Notice that $-1 \leq y \leq 1$ and so there exists $\theta \in [-\pi,\pi]$ such that $\sin(\theta) = y$. By the periodicity of $\sin$, for each $n \in \N$, $\sin(\theta + 2\pi n) = y$. Let $x_n = \frac{1}{\theta + 2\pi n}$. Then, $(x_n, \sin(1/x_n))$ is a sequence  of points in $S$. As $x_n \to 0$ as $n \to \infty$ and each $\sin(1/x_n) = y$, the limit of $(x_n, \sin(1/x_n))$ is $(0,y)$. Therefore, $Y \sq \ol{S}$ meaning that $X \sq \cl{S}$. Since $\ol{S} \sq X$ always, it follows  that $\ol{S} = X$, as desired.
        \end{proof}
    \end{claim}

    \begin{claim}
        $S$ is connected.

        \begin{proof}
            For any two points in $S$, the graph of $f(x) = \sin(1/x)$ provides a path between the two points. Therefore $S$ is path-connected. Since any path-connected set is also connected, $S$ is connected.
        \end{proof}
    \end{claim}

    Since $S \sq X \sq \cl{S}$ and $S$ is connected, $X$ must be connected (\ref{thm:clConnected}).\\

    Seeking a contradiction, suppose that $X$ is path-connected. Let $\theta = 1/2\pi$, $x = (\theta,\sin(1/\theta)) \in S$ and $y = (0,0) \in Y$. Assume that $\gamma: [0,1] \to X$ is a path from $x$ to $y$. Then, $\gamma$ is a continuous map where $\gamma(0) = x$ and $\gamma(1) = y$. Let $\epsilon = \frac{1}{2}$ and since $\gamma$ is continuous there exists $\delta > 0$ where $t \in (1 - \delta, 1]$ implies that $\norm[\gamma(t) - \gamma(1)] < \epsilon$. That is, for each $t \in (1-\delta,1]$, $\gamma(t)$ is in the ball of radius 1/2 about the origin. Write $(x_0, y_0) = \gamma(1 - \delta)$. Let $p$ be the projection map of $\R^2$ onto the $x$-axis. Then, $f = p \circ \gamma$ is a composition of continuous maps and is therefore continuous. Notice that $0, x_0 \in f((1-\delta,1])$. Since continuous maps preserve connectedness, $f((1-\delta,1])$ is a connected subset of $\R$ that contains both $0$ and $x_0$. But the only connected sets in $\R$ are intervals and therefore $[0,x_0] \sq f((1-\delta,1])$. This is impossible as there exists $n \in \N$ such that $0 < \frac{1}{\pi/2 + 2\pi n} < x_0$ and $f\left(\frac{1}{\pi/2 + 2\pi n}\right) = 1 > 1/2$.

\end{proof}

\begin{prop}{}{}
    A locally path-connected, connected space $X$ is path-connected.
\end{prop}

\begin{prop}{}{clConnected}
    Suppose that $H$ is connected and $K$ is such that $H \sq K \sq \ol{H}$. Then, $K$ is connected.
\end{prop}

\begin{proof}
    Suppose that $U$ and $V$ are nonempty, open, disjoint sets such that $U \cup V = K$. Then, $U \cap H$ and $V \cap H$ are both open in $H$ with respect to the subspace topology. Since $U \cap H$ and $V \cap H$ are disjoint and $H$ is connected, either $H \sq U$ or $H \sq V$. Without loss of generality, assume $H \sq U$.

    \begin{claim}
       $\ol{H} \sq U$.

        \begin{proof}
            Suppose not. Then there exists a limit point $x \in V$ of $H$. Since $x$ is a limit point of $H$, every open set containing $x$ must intersect $H$. However, $V$ is an open set and since $V \cap U = \varnothing$, $V$ is disjoint from $H$.
        \end{proof}
    \end{claim}

    Since $K \sq \ol{H}$ and $\ol{H} \sq U$, $K \sq U$. This is a contradiction of the choice in $U$ and $V$.
\end{proof}

\begin{prop}{}{}
    A closed set is disconnected if and only if it is a union of disjoint, closed sets.
\end{prop}

\begin{proof}
    This follows quickly from the definition of disconnected. Assume that $X$ is the disjoint union $U \cup V$ with both sets nonempty and open. Then, $X \sm U$ and $X\sm V$ are disjoint, closed sets such that $X = (X \sm U) \cup (X \sm V)$.
\end{proof}

\begin{prop}{Heine Borel Theorem}{}
    Let $X \sq \R^n$. Then, $X$ is closed and bounded if and only if $X$ is compact.
\end{prop}

\begin{prop}{}{compactHaussNormal}
    Every compact Hausdorff space is normal.
\end{prop}

\begin{proof}
    Suppose that $X$ is a compact Hausdorff space. Let $A,B \sq X$ be nonempty, disjoint, closed sets. Notice that $A,B$ are both compact since they are closed subsets of a compact space (\ref{thm:compactClosed}).\\
    
    Fix a point $a \in A$. For each $b \in B$, choose disjoint open sets $U_{a,b}$ and $V_{a,b}$ such that $a \in U_{a,b}$ and $b \in V_{a,b}$. The collection $\{V_{a,b}\}$ forms an open cover for $B$ and since $B$ is compact, there exists a finite subcover, say $\{V_{a,b_1},\ldots, V_{a,b_n}\}$. Then the corresponding intersection $U_a = \bigcap_{k=1}^n U_{a,b_k}$ is an open set containing $a$ that is disjoint from $B$. Define $V_a = \bigcup_{k=1}^n V_{a,b_k}$. Then $U_a$ and $V_a$ are disjoint open sets.\\

    Repeat this process for each $a \in A$ to generate an open cover $\{U_a\}$ for $A$. Since $A$ is compact, there exists a finite subcover, say $\{U_{a_1},\ldots, U_{a_m}\}$. Let $U = \bigcup_{k=1}^m U_{a_k}$ and $V = \bigcap_{k=1}^m V_{a_k}$. Both $U$ and $V$ are open sets and by construction are disjoint such that $A \sq U$ and $B \sq V$.
\end{proof}

\begin{prop}{}{metNormal}
    Every metrizable space is normal.
\end{prop}

\begin{proof}
    Suppose that $X$ is a metrizable space and that $d$ is a metric on $X$. Let $A,B \sq X$ be closed and disjoint subsets. Define $f: X \to [0,1]$ by
        \[f(x) = \frac{d(x,A)}{d(x,A) + d(x,B)}.\]
    Here,
        \[d(x,A) = \inf_{y\in A} \{d(x,y)\}\]
    and $d(x,B)$ is defined similarly. Because $A$ and $B$ are closed, if $d(x,A) = 0$ then $x \in A$ and so $x \not\in B$ meaning that $d(x,B) > 0$. In particular this means that at most one of $d(x,A)$ and $d(x,B)$ can be zero and so $f$ is well-defined. For any $a \in A$, $f(a) = 1$ and for any $b \in B$, $f(b) = 0$.\\

    Since $f$ is the composition, quotient, and sum of continuous functions, $f$ is continuous. Therefore the sets
        \[U = f\inv\left([0,1/3)\right)\]
    and
        \[V = f\inv\left((2/3,1]\right)\]
    are open sets where $B \sq U$ and $A \sq V$.
\end{proof}

\begin{prop}{}{prodRegular}
    If $X$ and $Y$ are both regular, then $X \times Y$ is regular.
\end{prop}

\begin{proof}
    Consider the following lemma: 
    \begin{claim}
        A space $X$ is regular if and only if for each $x \in X$ and open neighborhood $U$ of $x$ there exists an open neighborhood $V$ of $x$ such that $x \in V \sq \ol{V} \sq U$.

        \begin{proof}
            Assume first that $X$ is regular. Let $x \in X$ and $U$ an open neighborhood of $x$. Define $C = X \sm U$. Then $x$ is a point and $C$ is a closed subset of $X$ that is disjoint from $x$. Since $X$ is regular, there exist disjoint open sets $V$ and $W$ containing $x$ and $C$ respectively. As $V$ and $W$ are disjoint, it follows that $\ol{V} \cap C = \varnothing$. That is, $\ol{V} \sq U$.\\

            Let $x \in X$ and let $E \sq X$ be a closed set with $x \not\in E$. Then, $X \sm E$ is an open neighborhood of $x$. By assumption, there exists an open neighborhood $V$ of $x$ such that $x \in V \sq \ol{V} \sq U$. Then $V$ is an open set containing $x$, $X - \ol{V}$ is an open set containing $E$, and $V \cap (X - \ol{V}) = \varnothing$.
        \end{proof}
    \end{claim}

    Let $(x,y) \in X \times Y$ and let $U \times V$ be a basic open neighborhood of $(x,y)$. Because $X$ is regular, there exists an open set $A \sq X$ such that $x \in A \sq \ol{A} \sq U$. Similarly, there exists an open set $B \sq Y$ such that $y \in B \sq \ol{B} \sq V$. Then $A \times B$ is an open set in $X \times Y$ such that $(x,y) \in A \times B \sq \ol{A \times B} = \ol{A} \times \ol{B} \sq U \times V$. By the lemma, this proves that $X \times Y$ is regular.
\end{proof}

\begin{prop}{}{}
    Let $X$ and $Y$ be topological spaces and suppose that $U,V \sq X$ and $W \sq Y$. Then,
    \begin{enumerate}[(a)]
        \item $\Int{U} \cap \Int{V} = \Int{U \cap V}$.
        \item $\Int{U} \cup \Int{V} \sq \Int{U \cup V}$.
        \item $\cl{U} \cup \cl{V} = \cl{U \cup V}$.
        \item $\cl{U} \cap \cl{V} \supseteq \cl{U \cap V}$.
        \item $X \sm \Int{U} = \cl{X \sm U}$.
        \item $X \sm \cl{U} = \Int{X \sm U}$.
        \item $\Int{U \times W} = \Int{U} \times \Int{W}$.
        \item $\cl{U \times W} = \cl{U} \times \cl{W}$.
    \end{enumerate}
\end{prop}


\end{document}