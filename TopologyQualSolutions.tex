\documentclass{article}

\usepackage[margin=1in]{geometry} 
\usepackage{amsmath,amsthm,amssymb, graphicx, multicol, array}
\usepackage{enumerate}
\usepackage{mathrsfs,xcolor} 
\usepackage[shortlabels]{enumitem}
\usepackage[utf8]{inputenc}
\usepackage{tcolorbox}
\tcbuselibrary{theorems}
\usepackage{mathtools}
\usepackage{dsfont}
\usepackage{graphicx}
\usepackage{float}
\usepackage{esint}
\setlength{\parindent}{0pt} % Don't indent new paragraphs
\setlength{\headheight}{24pt} 

\DeclareMathOperator{\inter}{int}


% To allow for referencing
\usepackage{nameref}
\usepackage{hyperref}



%%%%%%%%%%%%%%%%%%%%%%%%%%%%%%%%%%%%%%
% Commonly used commands shortcuts
\newcommand{\N}{\mathbb{N}}
\newcommand{\Z}{\mathbb{Z}}
\newcommand{\R}{\mathbb{R}}
\newcommand{\Q}{\mathbb{Q}}
\newcommand{\C}{\mathbb{C}}
\newcommand{\F}{\mathbb{F}}
\newcommand{\inv}{^{-1}}
\newcommand{\Aut}{\textrm{Aut}}
\newcommand{\Inn}{\textrm{Inn}}
\newcommand{\Int}{\textrm{int}}
\newcommand{\ph}{\varphi}
\newcommand{\epsi}{\epsilon}
\newcommand{\incl}{\xhookrightarrow{}}
\newcommand{\ident}{\mathds{1}}
\newcommand{\sq}{\subseteq}
\newcommand{\scr}{\mathscr}
\newcommand{\lcm}{\textrm{lcm}}
\newcommand{\sm}{\setminus}
%%%%%%%%%%%%%%%%%%%%%%%%%%%%%%%%%%%%%%
% Template Information
\newcommand{\class}{Topology Qualifying Exam}

% Make header with name and date etc.
\usepackage{fancyhdr}
\lhead{Topology Qualifying Exam}
\pagestyle{fancy}

%%%%%%%%%%%%%%%%%%%%%%%%%%%%%%%%%%%%%%
% Creating solution environment
\newenvironment{solution}
  {\renewcommand\qedsymbol{}\begin{proof}[Solution]}
  {\end{proof}}
%%%%%%%%%%%%%%%%%%%%%%%%%%%%%%%%%%%%%%

%%%%%%%%%%%%%%%%%%%%%%%%%%%%%%%%%%%%%%
% Customize theorem, lemma, and definition environments
\theoremstyle{definition}
\newtheorem{defn}{Definition}[section]

\newtcbtheorem[use counter*=defn]{thm}{Theorem}%
{fonttitle=\bfseries}{thm}

\newtcbtheorem[use counter*=defn]{lem}{Lemma}%
{fonttitle=\bfseries}{lem}

\newtcbtheorem[use counter*=defn]{cor}{Corollary}%
{fonttitle=\bfseries}{cor}

\newtcbtheorem[use counter*=defn]{prob}{Problem}%
{fonttitle=\bfseries}{prob}


\begin{document}

\section{Directory}

\begin{itemize}
\item \textbf{Fall 2020}: 1 (Problem \ref{prob:F20.1}), 
				2 (Problem \ref{prob:F20.2}), 
				3 (Problem \ref{prob:F20.3}), 
				4 (Problem \ref{prob:F20.4}),  
				5 (Problem \ref{prob:F20.5}), 
				6 (Problem \ref{prob:F20.6}), 
				7 (Problem \ref{prob:F20.7})
\item Fall 2016: 1 (Problem \ref{prob:F161}), 4 (Problem \ref{prob:F164})
\end{itemize}

\section{Point Set Properties}

	\begin{prob}{F20}{F20.1}
	\begin{enumerate}[(a)]
	\item Give an example of two topological spaces $X,Y$ and a continuous bijection $f: X \to Y$ that is not a homeomorphism.
	\item Show that if $X$ is compact and $Y$ is Hausdorff, then every continuous bijection between the spaces is a homeomorphism.
	\end{enumerate}
	\end{prob}
	
	\begin{solution}
	Let $X = [0,1]$ with the standard topology and $Y = [0,1]$ with the trivial topology. Let $f: X \to Y$ be the identity map. Clearly $f$ is bijective. The only open sets in $Y$ are $\varnothing $ and $[0,1]$. Since both $f\inv(\varnothing) = \varnothing$ and $f\inv([0,1]) = [0,1]$ are open in $X$, $f$ is continuous. However, $f$ is not a homeomorphism since $(0,1)$ is open in $X$ but $f(0,1) = (0,1)$ is not open in $Y$.
	\end{solution}
	
	\begin{proof}
	Let $f: X \to Y$ be a continuous bijection from a compact space to a Hausdorff space. To show that $f$ is a homeomorphism, it remains to check that $f$ is an open mapping. This is equivalent to proving that $f$ maps closed sets to closed sets. Let $A \sq X$ be a closed set. Since $X$ is compact, $A$ is compact in $X$. Then, $f(A) \sq Y$ must be compact since $f$ is continuous. In a Hausdorff space, any compact set is closed and thus $f(A)$ is closed in $Y$, as desired.
	\end{proof}

\section{Homeomorphic Spaces}

	\begin{prob}{S20}{}
	Prove that $S^2$ is homeomorphic to a quotient space of $S^1 \times [0,1]$.
	\end{prob}
	
	\begin{proof}
	Define an equivalence relation $\sim$ on $S^1 \times [0,1]$ such that
		\[(\theta,0) \sim (\theta', 0) \]
	and
		\[(\theta,1) \sim (\theta', 1) \]
	for any $\theta,\theta' \in S^1$. Then $S^1 \times [0,1]/ \sim$ is an annulus with each of the boundary disks crushed to a point. Note that
		\[S^2 = \{(\theta, \phi): 0 \leq \theta \leq 2\pi, 0 \leq \varphi \leq \pi \}. \]
	where all points of the form $(\theta, 0)$ correspond to the north pole of $S^2$ and all points of the form $(\theta, \pi)$ correspond to the south pole of $S^2$. Every other point in $S^2$ has a unique description in this coordinate system.\\
	
	Define $f: S^1 \times [0,1]/ \sim \to S^2$ by $f(\theta,t) = (\theta, \pi t)$. Observe that $f$ is well-defined as all points in $S^1 \times \{0\}$ are mapped to the north pole and all points in $S^1 \times \{1\}$ are mapped to the south pole. As both component functions of $f$ are continuous, $f$ is continuous. Given any $(\theta, \varphi) \in S^2$, $f(\theta, \varphi/\pi) = (\theta, \varphi)$, proving that $f$ is surjective. To see that $f$ is injective, suppose that $f(\theta,t) = f(\theta', t')$. Then, $(\theta,\pi t) = (\theta',\pi t')$. This means that $t = t'$. If $t = 0$, then $(\theta,0) \sim (\theta' 0)$. If $t = 1$, $(\theta,1) \sim (\theta' 1)$. If $t,t' \not\in \{0,\pi\}$ then $\theta = \theta'$. In any case, $(\theta,t) = (\theta, t') \in S^1 \times [0,1] /\sim$. As $f$ is a continuous bijection from a compact space to a Hausdorff space, $f$ is a homeomorphism.
	\end{proof}
	
\section{Metric Spaces}

	\begin{prob}{S20}{}
	Let $(X,d)$ be a metric space and fix a point $x_0 \in X$. Let $\rho$ be a new metric given by $\rho(x,y) = d(x,x_0) + d(y,x_0)$ whenever $x \neq y$ and $\rho(x,y) = 0$ if $x = y$. Verify that $\rho$ is a metric and $(X,\rho)$ is complete.
	\end{prob}
	
	\begin{proof}
	By construction, $\rho(x,y) \geq 0$ for each $x,y \in X$. Suppose $\rho(x,y) = 0$ but $x \neq y$. Then, $0 = \rho(x,y) = d(x,x_0) + d(y,y_0)$. Since at most one of $x$ and $y$ can be $x_0$, $d(x,x_0) + d(y,y_0) > 0$. Therefore $\rho(x,y) = 0$ if and only if $x = y$. Suppose now that $x,y, z \in X$. Then,
		\[\rho(x,y) + \rho(y,z) = d(x,x_0) + d(y,x_0) + d(y,x_0) + d(z,x_0) = \rho(x,z) + 2 d(y,x_0) \geq \rho(x,z) \]
	proving that $\rho$ is a metric.\\
	
	To see that $(X, \rho)$ is a complete metric space, let $(x_n)$ be a Cauchy sequence in $(X, \rho)$. Let $\epsilon > 0$ and choose $N \in \N$ sufficiently large such that $\rho(x_N, x_n) < \epsilon$ whenever $n \geq N$. This means that whenever $n \geq N$,
		\[ d(x_n,x_0) \leq d(x_N, x_0) + d(x_n, x_0) = \rho(x_N, x_m) < \epsilon. \]
	Therefore, $x_n \to x_0$ in $(X,d)$. Equivalently, as $n \to \infty$, $d(x_n,x_0) \to 0$. Then,
		\[\rho(x_n, x_0) = d(x_n, x_0) + d(x_0, x_0) = d(x_n, x_0) \]
	meaning that as $n \to \infty$, $\rho(x_n,x_0) \to 0$. That is, $x_n \to x_0$ in $(X,\rho)$.
	\end{proof}


\section{Fundamental Group}

	\begin{prob}{F20}{F20.4}
	Prove that no pair of the following spaces are homeomorphic to one another:
		\[ S^0, \; S^1 \times \R, \; S^1 \times S^2, \; \R \times S^2, \; S^2 \]
	\end{prob}
	
	\begin{proof}
	First note that $S^0$ is a discrete space while the remaining spaces are not. Therefore, $S^0$ cannot be homeomorphic to any of the other spaces. Because $S^1 \times \R$ and $\R \times S^2$ are unbounded and therefore not compact, neither of these spaces is homeomorphic to either of compact spaces, $S^1 \times S^2$ or $S^2$. As $S^1 \times \R$ is the product of path-connected spaces, $\pi_1(S^1 \times \R) \cong \pi_1(S^1) \times \pi_1(\R) \cong \Z$. Similarly, $\pi_1(\R \times S^2) \cong \pi_1(\R) \times \pi_1(S^2) \cong 0$. As the fundamental group is preserved under homeomorphisms, $S^1 \times \R$ and $\R \times S^2$ are not homeomorphic. Similarly, $S^1 \times S^2$ and $S^2$ are not homeomorphic since $\pi_1(S^1 \times S^2) \cong \Z$ and $\pi_1(S^2) = 0$.
	\end{proof}
	
\section{Unfinished}

	\begin{prob}{F20}{F20.7}
	A subset $E$ of a topological space $X$ is called a $G_\delta$ if there is a sequence $U_1, U_2, \ldots$ of open subsets of $X$ such that $E = \cap_{j}U_j$.
	\begin{enumerate}[(i)]
	\item Show that if $f: X \to \R$ is a continuous function from $X$ to the real line, then $\{x: f(x)=0\}$ is closed and is a $G_\delta$.
	\item Show that in a metric space, every closed set is a $G_\delta$.
	\item Prove  that $(ii)$ fails in an arbitrary topological space.
	\end{enumerate}
	\end{prob}

	\begin{prob}{F16}{F161}
	Give a proof or counter example for the following:
	\begin{enumerate}[(a)]
	\item Every closed subset of a compact space is compact.
	\item The product of any two connected spaces is connected.
	\end{enumerate}
	\end{prob}
	
	\begin{prob}{F16}{F164}
	A topological space $X$ is \emph{regular} is for every closed subset $C$ of $X$ and point $p \in X \sm C$, there are disjoint open sets $U, V \sq X$ with $C \sq U$ and $p \in V$. Prove that every compact Hausdorff space is regular.
	\end{prob}
	
	\begin{prob}{S20}{}
	Prove that the product of two regular spaces is regular.
	\end{prob}
	
	\begin{prob}{S20}{}
	A topological space is called \emph{totally disconnected} if every pair of points is contained in a pair of disjoint open sets whose union is the whole space. Prove that every countable metric space is totally disconnected.
	\end{prob}
	
	\begin{prob}{F16}{}
	Give an example of a space that is connected but not path-connected. Prove the example works.
	\end{prob}
	\begin{prob}{F16}{}
	Prove that a metric space is compact if and only if it is sequentially compact.
	\end{prob}
	
	\begin{prob}{S20}{}
	Let $X$ be a compact metric space. Prove that there exists a finite set of points $x_1, \ldots, x_n$ such that every point in $X$ is distance less than 3 from some $x_i$ and $d(x_i,x_j) \geq 1$ for any $i \neq j$.
	\end{prob}
	
	\begin{prob}{S20}{}
	Suppose that $X$ is a metric space such that every sequence in $X$ has a Cauchy subsequence. Prove that $X$ can be covered by finitely many balls of radius 1.
	\end{prob}
	
	\begin{prob}{F20}{F20.3}
	Let $(X,d)$ be a metric space and let $f: X \to X$ be a continuous function without any fixed points.
	\begin{enumerate}[(i)]
	\item If $X$ is compact, show that there exists $\epsilon > 0$ so that $d(x,f(x)) > \epsilon$ for all $x \in X$.
	\item Show that this fails if $X$ is not compact.
	\end{enumerate}
	\end{prob}
	
	\begin{prob}{F16}{}
	For each of the following either give a proof or provide a justified counterexample.
	\begin{enumerate}[(a)]
	\item Suppose that $A$ and $B$ are non-empty topological spaces and $A \times B$ is equipped with the product topology. Let $\sim$ be the equivalence relation on $A \times B$ defined by $(a,b) \sim (a', b')$ if and only if $b = b'$. Is $A \times B/\sim$ homeomorphic to $A$?
	\item Suppose that $B$ and $C$ are subspaces of a topological space $A$. If $B$ is homeomorphic to $C$, does it follow that $A/B$ is homeomorphic to $A/C$?
	\end{enumerate}
	\end{prob}
		
	\begin{prob}{F16}{}
	State the contraction mapping theorem. Prove there is a unique continuous function $f: [0,1] \to [0,1]$ that satisfies
		\[f(x) = \frac{f(\sin x) + \cos x}{2} \]
	for all $x \in [0,1]$.
	\end{prob}
	
	
	\begin{prob}{S20}{}
	A topological space is \emph{separable} if it has a countable dense subset. Prove that the product of countable collection of separable topological spaces is separable.
	\end{prob}
	
	\begin{prob}{F20}{F20.2}
	Let $X$ be a topological space. Show that the intersection of any two dense open sets in $X$ is also dense. Give an example that shows that this may fail if the two sets are not required to be open.
	\end{prob}
	
	
	
	\begin{prob}{F20}{F20.5}
	\begin{enumerate}[(i)]
	\item Suppose that $X$ is a topological space with the property that every two point space lies in a connected subspace of $X$. Prove that $X$ is connected.
	\item Suppose that the work \textbf{TOPOLOGY} is written in purple ink on a square of white paper. Let $V$ denote the subspace consisting of the white paper that remains. How many path-connected components  does $V$ have? For each such component $X$, compute $\pi_1(X)$.
	\end{enumerate}
	\end{prob}
	
	\begin{prob}{F20}{F20.6}
	Suppose that $X$ is a metric space. Define what it means for $C \sq X$ to be \emph{complete}.
	\begin{enumerate}[(i)]
	\item Show that if $C$ and $D$ are complete subsets of $X$ then $C\cup D$ is complete.
	\item Suppose that $\{C_\lambda\}$ is a family of complete subspaces of $X$. Prove that $\cap_{\lambda} C_{\lambda}$ is either empty or complete.
	\end{enumerate}
	\end{prob}

	
	\begin{prob}{F19}{}
	Give careful definitions of \emph{continuity} and \emph{uniform continuity} for maps between metric spaces.
	\begin{enumerate}[(i)]
	\item Show that if $f: X \to Y$ is a continuous map between metric spaces and $X$ is compact, then $f$ is uniformly continuous.
	\item Prove or disprove: If $f: X \to Y$ is a uniformly continuous map between metric spaces and $X$ is complete, then $Y$ is complete.
	\end{enumerate}
	\end{prob}
	
	\begin{prob}{F19}{}
	Let $X$ be the set of subsets of $\N$. If $A$ is a finite subset of $\N$ and $B$ is a subset of $\N$ whose complement is finite, define a subset $[A,B]$ of $X$ by
		\[ [A,B] = \{E \sq \N: A \sq E \sq B \} \]
	Show that the sets $[A,B]$ form a base for a topology on $X$. Prove that with this topology, $X$ is Hausdorff and disconnected. Prove that the function $f:X \times X \to Y$ given by
		\[ f(E_1, E_2) = E_1 \cap E_2 \]
	is continuous.
	\end{prob}
	
	\begin{prob}{F19}{}
	Are the following true or false? Give a proof or counter-example.
	\begin{enumerate}[(a)]
	\item If $X = U \cup V$ where $U$ and $V$ are both open and simply connected, then $X$ is simply connected.
	\item If $f: X \to Y$ is a continuous map which is onto, then $f_*: \pi_1(X) \to \pi_1(Y)$ is onto.
	\item If $f: X \to Y$ is a continuous map which is injective, then $f_*: \pi_1(X) \to \pi_1(Y)$ is injective.
	\end{enumerate}
	\end{prob}
	
	\begin{prob}{F19}{}
	Given $\epsilon>0$, two points $a,b$ of a metric space $M$ are said to be \emph{connected by an $\epsilon$-chain}, if there exist points $x_0, \ldots, x_n \in M$ such that $x_0 = a$, $x_n = b$ and $d(x_i, x_{i+1}) < \epsilon$ for each $i = 0, \ldots, n-1$.
	\begin{enumerate}[(a)]
	\item Show that if $M$ is connected, then for every $\epsilon > 0$ any two points are connected by an $\epsilon$-chain. Provide an example to show that the converse does not hold.
	\item Show that if $M$ is a compact metric space and for every $\epsilon > 0$ any two points  of $M$ are connected by an $\epsilon$-chain, then $M$ is connected.
	\end{enumerate}
	\end{prob}
\end{document}
